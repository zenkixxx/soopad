%%  JLG input 2/18/2017 
%% TGARS Draft - JLG input 8/11/2016
%%bare_jrnl.tex
%% V1.4a
%% 2014/09/17
%% by Michael Shell
%% see http://www.michaelshell.org/
%% for current contact information.
%%
%% This is a skeleton file demonstrating the use of IEEEtran.cls
%% (requires IEEEtran.cls version 1.8a or later) with an IEEE
%% journal paper.
%%
%% Support sites:
%% http://www.michaelshell.org/tex/ieeetran/
%% http://www.ctan.org/tex-archive/macros/latex/contrib/IEEEtran/
%% and
%% http://www.ieee.org/

%%************************************************************************
%% Legal Notice:
%% This code is offered as-is without any warranty either expressed or
%% implied; without even the implied warranty of MERCHANTABILITY or
%% FITNESS FOR A PARTICULAR PURPOSE! 
%% User assumes all risk.
%% In no event shall IEEE or any contributor to this code be liable for
%% any damages or losses, including, but not limited to, incidental,
%% consequential, or any other damages, resulting from the use or misuse
%% of any information contained here.
%%
%% All comments are the opinions of their respective authors and are not
%% necessarily endorsed by the IEEE.
%%
%% This work is distributed under the LaTeX Project Public License (LPPL)
%% ( http://www.latex-project.org/ ) version 1.3, and may be freely used,
%% distributed and modified. A copy of the LPPL, version 1.3, is included
%% in the base LaTeX documentation of all distributions of LaTeX released
%% 2003/12/01 or later.
%% Retain all contribution notices and credits.
%% ** Modified files should be clearly indicated as such, including  **
%% ** renaming them and changing author support contact information. **
%%
%% File list of work: IEEEtran.cls, IEEEtran_HOWTO.pdf, bare_adv.tex,
%%                    bare_conf.tex, bare_jrnl.tex, bare_conf_compsoc.tex,
%%                    bare_jrnl_compsoc.tex, bare_jrnl_transmag.tex
%%*************************************************************************


% *** Authors should verify (and, if needed, correct) their LaTeX system  ***
% *** with the testflow diagnostic prior to trusting their LaTeX platform ***
% *** with production work. IEEE's font choices and paper sizes can       ***
% *** trigger bugs that do not appear when using other class files.       ***                          ***
% The testflow support page is at:
% http://www.michaelshell.org/tex/testflow/



\documentclass[draftcls,onecolumn]{IEEEtran}  % JLG -recommend one-column for drafts
%
% If IEEEtran.cls has not been installed into the LaTeX system files,
% manually specify the path to it like:
% \documentclass[journal]{../sty/IEEEtran}





% Some very useful LaTeX packages include:
% (uncomment the ones you want to load)


% *** MISC UTILITY PACKAGES ***
%
%\usepackage{ifpdf}
% Heiko Oberdiek's ifpdf.sty is very useful if you need conditional
% compilation based on whether the output is pdf or dvi.
% usage:
% \ifpdf
%   % pdf code
% \else
%   % dvi code
% \fi
% The latest version of ifpdf.sty can be obtained from:
% http://www.ctan.org/tex-archive/macros/latex/contrib/oberdiek/
% Also, note that IEEEtran.cls V1.7 and later provides a builtin
% \ifCLASSINFOpdf conditional that works the same way.
% When switching from latex to pdflatex and vice-versa, the compiler may
% have to be run twice to clear warning/error messages.






% *** CITATION PACKAGES ***
%
%\usepackage{cite}
% cite.sty was written by Donald Arseneau
% V1.6 and later of IEEEtran pre-defines the format of the cite.sty package
% \cite{} output to follow that of IEEE. Loading the cite package will
% result in citation numbers being automatically sorted and properly
% "compressed/ranged". e.g., [1], [9], [2], [7], [5], [6] without using
% cite.sty will become [1], [2], [5]--[7], [9] using cite.sty. cite.sty's
% \cite will automatically add leading space, if needed. Use cite.sty's
% noadjust option (cite.sty V3.8 and later) if you want to turn this off
% such as if a citation ever needs to be enclosed in parenthesis.
% cite.sty is already installed on most LaTeX systems. Be sure and use
% version 5.0 (2009-03-20) and later if using hyperref.sty.
% The latest version can be obtained at:
% http://www.ctan.org/tex-archive/macros/latex/contrib/cite/
% The documentation is contained in the cite.sty file itself.






% *** GRAPHICS RELATED PACKAGES ***
%
\ifCLASSINFOpdf
   \usepackage[pdftex]{graphicx}
  % declare the path(s) where your graphic files are
   \graphicspath{{../pdf/}{../jpeg/}}
  % and their extensions so you won't have to specify these with
  % every instance of \includegraphics
   \DeclareGraphicsExtensions{.pdf,.jpeg,.png}
\else
  % or other class option (dvipsone, dvipdf, if not using dvips). graphicx
  % will default to the driver specified in the system graphics.cfg if no
  % driver is specified.
  % \usepackage[dvips]{graphicx}
  % declare the path(s) where your graphic files are
  % \graphicspath{{../eps/}}
  % and their extensions so you won't have to specify these with
  % every instance of \includegraphics
  % \DeclareGraphicsExtensions{.eps}
\fi
% graphicx was written by David Carlisle and Sebastian Rahtz. It is
% required if you want graphics, photos, etc. graphicx.sty is already
% installed on most LaTeX systems. The latest version and documentation
% can be obtained at: 
% http://www.ctan.org/tex-archive/macros/latex/required/graphics/
% Another good source of documentation is "Using Imported Graphics in
% LaTeX2e" by Keith Reckdahl which can be found at:
% http://www.ctan.org/tex-archive/info/epslatex/
%
% latex, and pdflatex in dvi mode, support graphics in encapsulated
% postscript (.eps) format. pdflatex in pdf mode supports graphics
% in .pdf, .jpeg, .png and .mps (metapost) formats. Users should ensure
% that all non-photo figures use a vector format (.eps, .pdf, .mps) and
% not a bitmapped formats (.jpeg, .png). IEEE frowns on bitmapped formats
% which can result in "jaggedy"/blurry rendering of lines and letters as
% well as large increases in file sizes.
%
% You can find documentation about the pdfTeX application at:
% http://www.tug.org/applications/pdftex


\usepackage{diagbox}


% *** MATH PACKAGES ***
%
\usepackage[cmex10]{amsmath}
% A popular package from the American Mathematical Society that provides
% many useful and powerful commands for dealing with mathematics. If using
% it, be sure to load this package with the cmex10 option to ensure that
% only type 1 fonts will utilized at all point sizes. Without this option,
% it is possible that some math symbols, particularly those within
% footnotes, will be rendered in bitmap form which will result in a
% document that can not be IEEE Xplore compliant!
%
% Also, note that the amsmath package sets \interdisplaylinepenalty to 10000
% thus preventing page breaks from occurring within multiline equations. Use:
%\interdisplaylinepenalty=2500
% after loading amsmath to restore such page breaks as IEEEtran.cls normally
% does. amsmath.sty is already installed on most LaTeX systems. The latest
% version and documentation can be obtained at:
% http://www.ctan.org/tex-archive/macros/latex/required/amslatex/math/





% *** SPECIALIZED LIST PACKAGES ***
%
\usepackage{algorithmic}
% algorithmic.sty was written by Peter Williams and Rogerio Brito.
% This package provides an algorithmic environment fo describing algorithms.
% You can use the algorithmic environment in-text or within a figure
% environment to provide for a floating algorithm. Do NOT use the algorithm
% floating environment provided by algorithm.sty (by the same authors) or
% algorithm2e.sty (by Christophe Fiorio) as IEEE does not use dedicated
% algorithm float types and packages that provide these will not provide
% correct IEEE style captions. The latest version and documentation of
% algorithmic.sty can be obtained at:
% http://www.ctan.org/tex-archive/macros/latex/contrib/algorithms/
% There is also a support site at:
% http://algorithms.berlios.de/index.html
% Also of interest may be the (relatively newer and more customizable)
% algorithmicx.sty package by Szasz Janos:
% http://www.ctan.org/tex-archive/macros/latex/contrib/algorithmicx/




% *** ALIGNMENT PACKAGES ***
%
\usepackage{array}
% Frank Mittelbach's and David Carlisle's array.sty patches and improves
% the standard LaTeX2e array and tabular environments to provide better
% appearance and additional user controls. As the default LaTeX2e table
% generation code is lacking to the point of almost being broken with
% respect to the quality of the end results, all users are strongly
% advised to use an enhanced (at the very least that provided by array.sty)
% set of table tools. array.sty is already installed on most systems. The
% latest version and documentation can be obtained at:
% http://www.ctan.org/tex-archive/macros/latex/required/tools/


% IEEEtran contains the IEEEeqnarray family of commands that can be used to
% generate multiline equations as well as matrices, tables, etc., of high
% quality.

% *** SUBFIGURE PACKAGES ***
%\ifCLASSOPTIONcompsoc
%  \usepackage[caption=false,font=normalsize,labelfont=sf,textfont=sf]{subfig}
%\else
%  \usepackage[caption=false,font=footnotesize]{subfig}
%\fi
% subfig.sty, written by Steven Douglas Cochran, is the modern replacement
% for subfigure.sty, the latter of which is no longer maintained and is
% incompatible with some LaTeX packages including fixltx2e. However,
% subfig.sty requires and automatically loads Axel Sommerfeldt's caption.sty
% which will override IEEEtran.cls' handling of captions and this will result
% in non-IEEE style figure/table captions. To prevent this problem, be sure
% and invoke subfig.sty's "caption=false" package option (available since
% subfig.sty version 1.3, 2005/06/28) as this is will preserve IEEEtran.cls
% handling of captions.
% Note that the Computer Society format requires a larger sans serif font
% than the serif footnote size font used in traditional IEEE formatting
% and thus the need to invoke different subfig.sty package options depending
% on whether compsoc mode has been enabled.
%
% The latest version and documentation of subfig.sty can be obtained at:
% http://www.ctan.org/tex-archive/macros/latex/contrib/subfig/




% *** FLOAT PACKAGES ***
%
%\usepackage{fixltx2e}
% fixltx2e, the successor to the earlier fix2col.sty, was written by
% Frank Mittelbach and David Carlisle. This package corrects a few problems
% in the LaTeX2e kernel, the most notable of which is that in current
% LaTeX2e releases, the ordering of single and double column floats is not
% guaranteed to be preserved. Thus, an unpatched LaTeX2e can allow a
% single column figure to be placed prior to an earlier double column
% figure. The latest version and documentation can be found at:
% http://www.ctan.org/tex-archive/macros/latex/base/


%\usepackage{stfloats}
% stfloats.sty was written by Sigitas Tolusis. This package gives LaTeX2e
% the ability to do double column floats at the bottom of the page as well
% as the top. (e.g., "\begin{figure*}[!b]" is not normally possible in
% LaTeX2e). It also provides a command:
%\fnbelowfloat
% to enable the placement of footnotes below bottom floats (the standard
% LaTeX2e kernel puts them above bottom floats). This is an invasive package
% which rewrites many portions of the LaTeX2e float routines. It may not work
% with other packages that modify the LaTeX2e float routines. The latest
% version and documentation can be obtained at:
% http://www.ctan.org/tex-archive/macros/latex/contrib/sttools/
% Do not use the stfloats baselinefloat ability as IEEE does not allow
% \baselineskip to stretch. Authors submitting work to the IEEE should note
% that IEEE rarely uses double column equations and that authors should try
% to avoid such use. Do not be tempted to use the cuted.sty or midfloat.sty
% packages (also by Sigitas Tolusis) as IEEE does not format its papers in
% such ways.
% Do not attempt to use stfloats with fixltx2e as they are incompatible.
% Instead, use Morten Hogholm'a dblfloatfix which combines the features
% of both fixltx2e and stfloats:
%
% \usepackage{dblfloatfix}
% The latest version can be found at:
% http://www.ctan.org/tex-archive/macros/latex/contrib/dblfloatfix/




%\ifCLASSOPTIONcaptionsoff
%  \usepackage[nomarkers]{endfloat}
% \let\MYoriglatexcaption\caption
% \renewcommand{\caption}[2][\relax]{\MYoriglatexcaption[#2]{#2}}
%\fi
% endfloat.sty was written by James Darrell McCauley, Jeff Goldberg and 
% Axel Sommerfeldt. This package may be useful when used in conjunction with 
% IEEEtran.cls'  captionsoff option. Some IEEE journals/societies require that
% submissions have lists of figures/tables at the end of the paper and that
% figures/tables without any captions are placed on a page by themselves at
% the end of the document. If needed, the draftcls IEEEtran class option or
% \CLASSINPUTbaselinestretch interface can be used to increase the line
% spacing as well. Be sure and use the nomarkers option of endfloat to
% prevent endfloat from "marking" where the figures would have been placed
% in the text. The two hack lines of code above are a slight modification of
% that suggested by in the endfloat docs (section 8.4.1) to ensure that
% the full captions always appear in the list of figures/tables - even if
% the user used the short optional argument of \caption[]{}.
% IEEE papers do not typically make use of \caption[]'s optional argument,
% so this should not be an issue. A similar trick can be used to disable
% captions of packages such as subfig.sty that lack options to turn off
% the subcaptions:
% For subfig.sty:
% \let\MYorigsubfloat\subfloat
% \renewcommand{\subfloat}[2][\relax]{\MYorigsubfloat[]{#2}}
% However, the above trick will not work if both optional arguments of
% the \subfloat command are used. Furthermore, there needs to be a
% description of each subfigure *somewhere* and endfloat does not add
% subfigure captions to its list of figures. Thus, the best approach is to
% avoid the use of subfigure captions (many IEEE journals avoid them anyway)
% and instead reference/explain all the subfigures within the main caption.
% The latest version of endfloat.sty and its documentation can obtained at:
% http://www.ctan.org/tex-archive/macros/latex/contrib/endfloat/
%
% The IEEEtran \ifCLASSOPTIONcaptionsoff conditional can also be used
% later in the document, say, to conditionally put the References on a 
% page by themselves.




% *** PDF, URL AND HYPERLINK PACKAGES ***
%
%\usepackage{url}
% url.sty was written by Donald Arseneau. It provides better support for
% handling and breaking URLs. url.sty is already installed on most LaTeX
% systems. The latest version and documentation can be obtained at:
% http://www.ctan.org/tex-archive/macros/latex/contrib/url/
% Basically, \url{my_url_here}.




% *** Do not adjust lengths that control margins, column widths, etc. ***
% *** Do not use packages that alter fonts (such as pslatex).         ***
% There should be no need to do such things with IEEEtran.cls V1.6 and later.
% (Unless specifically asked to do so by the journal or conference you plan
% to submit to, of course. )

% correct bad hyphenation here
\hyphenation{op-tical net-works semi-conduc-tor}


\begin{document}
%
% paper title
% Titles are generally capitalized except for words such as a, an, and, as,
% at, but, by, for, in, nor, of, on, or, the, to and up, which are usually
% not capitalized unless they are the first or last word of the title.
% Linebreaks \\ can be used within to get better formatting as desired.
% Do not put math or special symbols in the title.
\title{Airborne Remote Sensing of Soil Moisture using P-band Signals of Opportunity}
%
%
% author names and IEEE memberships
% note positions of commas and nonbreaking spaces ( ~ ) LaTeX will not break
% a structure at a ~ so this keeps an author's name from being broken across
% two lines.
% use \thanks{} to gain access to the first footnote area
% a separate \thanks must be used for each paragraph as LaTeX2e's \thanks
% was not built to handle multiple paragraphs
%

\author{James~L. Garrison,~\IEEEmembership{Senior Member,~IEEE}
        Yao-Cheng~Lin,~\IEEEmembership{Member,~IEEE,},
        Jeffrey Piepmeier,~\IEEEmembership{Senior Member,~IEEE}
        Joesph Knuble,~\IEEEmembership{Senior Member,~IEEE}
        Georges~Stienne,~\IEEEmembership{Member,~IEEE,}
        and~Cornelis~Frederik~du~Toit,~\IEEEmembership{Senior Member,~IEEE}% <-this % stops a space
\thanks{Yao-Cheng Lin is with Aeronautics and Astronautics, Purdue University, Indiana,
IN, 47907}}

% note the % following the last \IEEEmembership and also \thanks - 
% these prevent an unwanted space from occurring between the last author name
% and the end of the author line. i.e., if you had this:
% 
% \author{....lastname \thanks{...} \thanks{...} }
%                     ^------------^------------^----Do not want these spaces!
%
% a space would be appended to the last name and could cause every name on that
% line to be shifted left slightly. This is one of those "LaTeX things". For
% instance, "\textbf{A} \textbf{B}" will typeset as "A B" not "AB". To get
% "AB" then you have to do: "\textbf{A}\textbf{B}"
% \thanks is no different in this regard, so shield the last } of each \thanks
% that ends a line with a % and do not let a space in before the next \thanks.
% Spaces after \IEEEmembership other than the last one are OK (and needed) as
% you are supposed to have spaces between the names. For what it is worth,
% this is a minor point as most people would not even notice if the said evil
% space somehow managed to creep in.



% The paper headers
\markboth{Journal of \LaTeX\ Class Files,~Vol.~13, No.~9, September~2014}%
{Shell \MakeLowercase{\textit{et al.}}: Bare Demo of IEEEtran.cls for Journals}
% The only time the second header will appear is for the odd numbered pages
% after the title page when using the twoside option.
% 
% *** Note that you probably will NOT want to include the author's ***
% *** name in the headers of peer review papers.                   ***
% You can use \ifCLASSOPTIONpeerreview for conditional compilation here if
% you desire.




% If you want to put a publisher's ID mark on the page you can do it like
% this:
%\IEEEpubid{0000--0000/00\$00.00~\copyright~2014 IEEE}
% Remember, if you use this you must call \IEEEpubidadjcol in the second
% column for its text to clear the IEEEpubid mark.



% use for special paper notices
%\IEEEspecialpapernotice{(Invited Paper)}




% make the title area
\maketitle

% As a general rule, do not put math, special symbols or citations
% in the abstract or keywords.
\begin{abstract}
Signals of Opportunity Airborne Demonstrator (SoOp-AD) is a NASA Instrument Incubator Program (IIP-13) selection to demonstrate remote sensing of Root Zone Soil Moisture (RZSM) using P-band signals of opportunity (SoOp). Soil penetration depth at these frequencies is 10’s of cm. Three basic observations will be produced from this instrument: autocorrelation of the direct signal, autocorrelation of the reflected signal, and the complex cross-correlation between the direct and reflected signals. Two reflectivity estimates, the ratio of the auto-correlations and the ratio of cross- to auto-correlation, have been defined, allowing soil moisture to be retrieved using established empirical models for the soil dielectric constant. In P-band, the available satellite transmissions have very low bandwidth (individual 5-25 kHz channels). An aircraft antenna designed for this wavelength would have a very wide beamwidth. These considerations preclude relying upon delay or antenna beam to provide strong isolation between direct and reflected ray paths. Consequently, both the sky-view and Earth-view antennas would receive a combination of direct and reflected signals. Algorithms for the retrieval of reflectivity, and thus RZSM, from observations of these signals must incorporate an understanding of this interference. Using synthetic signals having realistic noise power, a calibration function is developed to correct these observable s, accounting for the cross-channel interference. A sensitivity study was conducted, to evaluate the dependence of retrieval accuracy on the uncertainty in antenna gain pattern, aircraft attitude, and receiver calibration. These simulation studies, conducted over an ensemble of expected soil moisture, were used to define system requirements meeting the science requirement to retrieve volumetric soil moisture with a root mean square error of less than 0.04.
\end{abstract}

% Note that keywords are not normally used for peerreview papers.
\begin{IEEEkeywords}
Remote sensing, Signal of opportunity, soil moisture, P-band.
\end{IEEEkeywords}


% For peer review papers, you can put extra information on the cover
% page as needed:
% \ifCLASSOPTIONpeerreview
% \begin{center} \bfseries EDICS Category: 3-BBND \end{center}
% \fi
%
% For peerreview papers, this IEEEtran command inserts a page break and
% creates the second title. It will be ignored for other modes.
\IEEEpeerreviewmaketitle

\section{Introduction}
% The very first letter is a 2 line initial drop letter followed
% by the rest of the first word in caps.
% 
% form to use if the first word consists of a single letter:
% \IEEEPARstart{A}{demo} file is ....
% 
% form to use if you need the single drop letter followed by
% normal text (unknown if ever used by IEEE):
% \IEEEPARstart{A}{}demo file is ....
% 
% Some journals put the first two words in caps:
% \IEEEPARstart{T}{his demo} file is ....
% 
% Here we have the typical use of a "T" for an initial drop letter
% and "HIS" in caps to complete the first word.
[Jeff P ]
\IEEEPARstart{T}{he} measurement of soil moisture plays an important role in studying Earth's hydrosphere. Soil moisture is measured directly in situ at specific locations, however, the measurement in the region is not comprehensive for global studies because soil moisture highly depends on spatial distribution \cite{Wang:2009}. Satellite remote sensing provides several techniques for estimating soil moisture over a wide area, especially passive microwave radiometry is the most mature technology for the remote sensing of soil moisture. In 2009, ESA (European Space Agency) launched the SMOS (Soil Moisture and Ocean Salinity) satellite providing global maps of soil moisture and ocean salinity \cite{Kerr:2000}. Then the upcoming Soil Moisture Active Passive (SMAP) mission from National Aeronautics and Space Administration (NASA) \cite{Entekhabi:2010} will make global measurements of the soil moisture present at the Earth's land surface and will distinguish frozen from thawed land . However, the frequencies used in SMOS and SMAP are in L-band (1.4GHz) (the corresponding wavelength is about 21 cm) because of the required antenna size. Moreover, the specific frequency (1.4GHz) selected for SMOS and SMAP lies in a band protected for radio astronomy and thus utilizes strong isolation from radio frequency interference (RFI). The sensing depth for these frequencies, however, is limited to the first few centimeters of the soil.

Satellite remote sensing is not only limited in the observation of specific satellites, but Global Navigation Satellite System-Reflectometry (GNSS-R) has been demonstrated as an alternative approach to the remote sensing of soil moisture, through several airborne and ground-based experiments \cite{Zavorotny:2010} \cite{Larson:2008}. GNSS-R is a bistatic radar configuration, using the existing GNSS signals as the illumination source, producing a measurement of surface reflectivity, Γ, that is directly related to emissivity, ε, through conservation of energy; Γ=1-ε. GNSS-R thus provides a measurement of the exact physical property sensed by microwave radiometry, but with a substantially higher power due to the forward-scatter geometry and an active transmitter. This could enable a satellite receiver with a much smaller antenna, as compared to a radar or radiometer, to be used for soil moisture sensing. GNSS signals are designed with strong anti-jamming capabilities, a feature that should translate into a high RFI resistance for GNSS-R measurements, as compared to passive microwave radiometry. GNSS-R measurements from satellite would provide global coverage as a result of the GNSS constellation design. GNSS, however, is also an L-band system (1575.42 MHz) and thus, the observed depth of soil moisture would not be an improvement over that provided by L-band radiometry (a few cm).

\bf moved summary of GNSS-R here - backgrond information

\rm
GNSS-R is a bistatic radar configuration, using the existing GNSS signals as the illumination source, producing a measurement of surface reflectivity, $\Gamma$, that is directly related to emissivity, $\epsilon$, through conservation of energy; $ \Gamma = 1 - \epsilon $ \cite{Jin:2011}. GNSS-R thus provides a measurement of the exact physical property sensed by microwave radiometry, but with a substantially higher power due to the forward-scatter geometry and an active transmitter. This could enable a satellite receiver with a much smaller antenna, as compared to a radar or radiometer, to be used for soil moisture sensing. GNSS signals are designed with strong anti-jamming capabilities, a feature that should translate into a high RFI resistance for GNSS-R measurements, as compared to passive microwave radiometry. GNSS-R measurements from satellite would provide global coverage as a result of the GNSS constellation design. GNSS, however, is also an L-band system (1575.42 MHz) and thus, the observed depth of soil moisture would not be an improvement over that provided by L-band
radiometry (a few cm). The unique contribution of the research defined in this proposal would extend the penetration depth of soil moisture measurements by applying the GNSS-R technique to communication satellite signals at lower frequencies.

To observe the soil moisture of different depths, the signals with various frequencies are necessary. Reflectometry, using these so-called "Signals of Opportunity", or SoOp, could potentially make these measurements with instrumentation that is an order of magnitude smaller and with lower power requirement than active SAR or passive radiometry. As with GNSS-R, the "process-gain" obtained through cross-correlating the reflected signal may also provide a stronger isolation from RFI. Such measurements have never been demonstrated before. The cross-correlation technique developed for GNSS-R was recently demonstrated on communication satellite signals. An important finding was that the higher SNR, and nearly random data modulation, of communications signals allowed cross-correlation to be performed between the direct and reflected signals, without the need for a model signal. In that experiment, ocean reflections from the two geostationary satellites providing the XM radio service were inverted to estimate the surface wind speed during a flight off the Chesapeake Bay \cite{Shah:2011}. According to successful demonstration to show reflectometry using any communication satellite signal with sufficient SNR and a nearly random modulation, the similar technology is applied to the other satellites such as UHF-follow On (UFO) satellites with frequency from 240 to 270 MHz.


% You must have at least 2 lines in the paragraph with the drop letter
% (should never be an issue)

% needed in second column of first page if using \IEEEpubid
%\IEEEpubidadjcol


\section{Basic Principles}

[JLG - here the basic physics:  soil moisture to reflectivity, to correlation, details of the signal model for the instrument come later]

According to statement of energy conversation, the relationship between reflectivity $\Gamma$ and emissivity $T$ is expressed as $\Gamma + T = 1$. Reflectivity, $\Gamma$ can be used to directly sense the soil moisture through the   Fresnel  coefficients $\mathcal{R}$, derived for the specular reflection from the interface between two semi-infinite media.
$\Gamma_{pq} = |\mathcal{R}_{pq}|^2$ 
For an incident polarization $p$ and reflected polarization $q$. Circular polarization is used for most satellite communication transmissions.  The Fresnel coeffecients for incident right (R) and left(L) hand circular polarization are derived from the common horizontal (H) and Vertical (V) polarization forms. 

\begin{equation}
	\begin{bmatrix}
	\mathcal{R}_{RR} & \mathcal{R}_{RL} \\
	\mathcal{R}_{LR} & \mathcal{R}_{LL}
	\end{bmatrix} = 
	\frac{1}{2}
	\begin{bmatrix}
	\mathcal{R}_{HH} + \mathcal{R}_{VV} & \mathcal{R}_{HH} - \mathcal{R}_{VV} \\
	\mathcal{R}_{HH} - \mathcal{R}_{VV} & \mathcal{R}_{HH} + \mathcal{R}_{VV}		\end{bmatrix}
\end{equation}

\begin{equation} 
\begin{split}
	{\mathcal{R}_{hh}} = {}& \frac{{{{\tilde \varepsilon }_r}\cos \theta  - \sqrt 				{{{\tilde \varepsilon }_r} - {{\sin }^2}\theta } }}{{{{\tilde \varepsilon}_r}				\cos \theta  + \sqrt {{{\tilde \varepsilon }_r} - {{\sin }^2}\theta }}}
\\
	{\mathcal{R}_{vv}} = {}& \frac{{\cos \theta  - \sqrt {{{\tilde \varepsilon }_r} - {{\sin }^2}\theta } }}{{\cos \theta  + \sqrt {{{\tilde \varepsilon }_r} - {{\sin }^2}\theta }}} 
    \end{split}
     \label{Eq: reflectivity_di}
\end{equation}

Where $\tilde{ \varepsilon }_r$ is the dielectric constant of soil, and  $\theta$ is the elevation angle. The dielectric constant of soil depends on the soil moisture, with several models developed and in use for remote sensing \cite{Wang:2009}. In this paper, we will present the essential design of an instrument for this measurement, using reflections of P-band signals of opportunity (SoOp) to estimate the surface reflectivity $\Gamma$. 
An approach to estimating the soil reflectivity through cross-correlating the reflected signal with the direct one will be developed in section \ref{section:Algorithm}.  Inverse methods to estimate the water distribution throughout a soil column from these measurements are beyond the scope of the work in this paper.  Figure \ref{fig:reflectivity} shows the reflectivity for linear and circular polarization as a function of volumentric soil moisture, for several common soil types (sandy clay, loam, and silty clay loam), computed using the model for complex dielectric constant  \cite{Peplinski:1995} \cite{Peplinski_correct:1995}.   
\begin{figure}[t!]
	\centering
	\includegraphics[width=4in]{pdf/gamma_vs_mv3.pdf}
	\caption{The reflectivity vs. soil moisture VWC($\%$) for the circular polarization with different soil types, 260 MHz frequency, and 20 $^\circ C$ temperature}
	\centering
	\label{fig:reflectivity}
\end{figure}
In figure \ref{fig:reflectivity}, the soil textures  of soil 1 to 3 are  (50,40), (40,20), and (10,35) for the percentile of sand and clay, and they are plotted in blue, yellow, and green, respectively. The solid line is cross-polarization, and the dash line is co-polarization.

This relationship will be the fundamental approach to sensing the soil moisture content within approximately the top layers of soil down to the penetration depth given in figure \ref{fig:depth}. The Penetration depth depth within an absorbing homogeneous material, in which the flux  of the field decays to $\frac{1}{e}$ of its surface value \cite{Ulaby:1981}.  This can be shown to be dependent on the imaginary part of the soil dielectric constant and the wavelength $\lambda$.
\begin{equation}
  \delta_p  = \frac{\pi}{\lambda} Im(\sqrt{\tilde{\varepsilon}_{r}})
\end{equation}
Where  $\alpha$ attenuation constant. Figure \ref{fig:depth} plots the penetration depth as a function of volumetric soil moisture, computed using the model \cite{Peplinski:1995}, for three frequencies, representing P-band (260 MHz), L-band(1.57542 GHz) and S-band (2.342205 GHz). Although the situation in a realistic  natural environment will be more complicated, with a heterogeneous multi-layer soil, and variable soil moisture profile, this simple definition of penetration depth will be used as an approximation of the soil depth to which microwave remote sensing is capable of making a measurement.  Figure \ref{fig:depth} clearly shows that measurements below the top few cm. of soil will require observations in P-band and lower frequencies. 
\begin{figure}[!t]
	\centering
	\includegraphics[width=4in]{pdf/penetration_depth.pdf}
	\caption{Penetration depth with different frequencies, 20 $^\circ C$ temperature, and soil texture (40, 20); blue, red, and yellow lines shows the penetration depth with 260 MHz, 1.57542 GHz, and 2.342205 GHz, respectively.}
   \centering
	\label{fig:depth}
\end{figure}

Assuming a 45 deg. incidence angle and a 1.2~m wavelength,  P-band signals are expected to meet the Rayleigh criterion for specular reflection when soil height variation is smaller than 21 cm. The more stringent Fraunhofer condiiton \cite{Ulaby:1981} will be met for soil roughness smaller than 5 cm.  These are reasonable conditions to meet for agricultural fields, and we will proceed with these assumptions.  Validation of this assumption will be a key step in the forthcoming experimental tests of the Signals of Opportunity Airborne Demonstrator (SoOp-AD). 
The first Fresnel zone, assuming a distant transmitter, will be an ellipse with a semi-major axis  
\begin{equation}
	%F_n=\sqrt{n \lambda D_{RS}} \Rightarrow 
    a =\sqrt{\frac{\lambda h}{sin\theta} } 
     \label{Eq: Fresnel_zone_reduce}
\end{equation}
on the soil surface. $h$ is the height and $\theta$ is the elevation angle. The first fresnel zone will be used as the definition of the resolution of the specular reflectometry measurmenets. 

The transmitted signal from a satellite ($x_{T}$) is modeled as
\begin{equation}
	x_T(t)=\sqrt{C_T}a(t)e^{j\omega_et},
    \label{Eq:xT}
\end{equation}
where $a(t)$ is the baseband digital signal.  The received signals at the object with antenna along the direct and reflected path ray can be represented 
\begin{eqnarray}
	%\begin{split}
  	x_D(t)&=\sqrt{C_D}a(t-\tau_D)e^{j\omega_e(t-\tau_D)}, \\
    x_R(t)&=\sqrt{C_R}a(t-\tau_R)e^{j\omega_e(t-\tau_R)}, 
   % \end{split}
    \label{Eq:xD_xR}
\end{eqnarray}
where $C_D$ and $C_R$ are received powers of signals the direct and reflected path at the antennas.  
$\tau_D$ and $\tau_R$ are the path delay along the direct and reflected path. The reflectivity $\Gamma$ is contained within the ratio of the direct and reflected powers
\begin{equation}
	\Gamma=\frac{C_{R}}{C_{D}}
    \label{Eq:reflect}
\end{equation}
Although a straighforward principle, the implementation of an instrument to extract this information introduces some complications.  
Specifically, the SoOp receiver must perform the following key functions to get to the reflectivity as defined in \ref{Eq:reflect}: 1.) isolate the signal $x_R(t)$ from $x_D(t)$, 2.) Remove the modulation $a(t)$ containing the data, and 3.) Remove effects of the antenna and receiver gains, and system noise.  
The next section will define the instrument design, and explain these functions in principle.  
The following section (\ref{sec:algorithm}) will develop the theory supporting this design and the algorithm to extract reflectivity. 
Finally, section (\ref{sec:error}) presents a simulation and error analysis. 


\section{Instrument Design}

[Jeff P, Joe K]
\label{sec:instrument}
JLG: Do we want this here ? or after the theory section ? My view is give a general overview and then provide the numbers developed in the theory. 

\subsection{Requirements}

Host on B-200 or similar aircraft

Observation of geostationary transmitters operating between XXX and XXX deg. Elevation, transmitting between 230 and 270 MHZ, with bandwidth of 5-25 KhZ, separated by XXX KHz. [give general overview here, but state the key assumptions about the transmitted source which drives the design process]  A detailed description of the signal source is provided in \ref{app:signal_source}

Working requirement for end-to-end retrieval accuracy of 0.04 volumetric water content.

Precision for altitude and attitude, which map through antenna gain. 

Table \ref{tab:sysreq} summarizes the system requirements derived from the science requirements [ jlg do we want to break these down so that you can show the flow-down ? or just state the key driving requirements]

Error analysis in \ref{sec:error} will show that these working requirements are expected to produce a system meeting the retrieval precision requirements. 

\begin{table}
\label{tab:sysreq}
\caption{System Requirements}
\begin{tabular}{|c|c|c|}
\hline
Parameter &  Value & Units \\
\hline
VMC Retrieval Precision &  0.04 & $m^3/m^3$ \\
\hline
Altitude precision &  100 & m \\
\hline
Attitude precision & 0.1 & deg \\
\hline 
Antenna Gain Knowledge & 0.2 & dB \\
\hline 
\end{tabular}

\end{table}

A summary of the complete system is shown in figure \ref{fig:schematic}.  It consists of a pair of two dual-pol antennas elements, identified as ``A'' and ``B'', connected to two parallel and independent analog sections, identified as ``1'' and ``2''  through a transfer switch.  The transfer switch will enable calibration of the receiver gain variations.  An antenna beam forming technique (described in \ref{sec:antenna}) will combined signals A and B to form a sky-view (S) and Earth-view (E) beam.  
The design requirement for this is to focus the S beam in the direction of the direct (D) line of sight signal with a null in the direction of the reflected (R) signal. 
Similarly, the E beam will be formed to have a maximum gain in the R signal direction, with a null in the D signal direction.
This beam-forming is not perfect, and quantifying the residual signal from R and D arriving in the S and E beams, respectively, will form an important part of the system error budget, developed in section \ref{sec:error}. 
Additional capability for calibration is provided by a calibrated thermal noise source.
A digital section will cross-correlate the the signals from receivers 1 and 2, at various fixed lags, to generate an array of complex cross-correlation values, which form the instrument or Level 0 data products.


Although assumed identical in design, the two receivers have different gains, $G_1$ and $G_2$, and different noise figures, both of which must be calibrated using a process described in section \ref{sec:calibration}.  
A transfer switch, which changes the connections between the antennas $\{ E, S \}$ and the receiver channels, $\{ 1, 2\}$ and noise sources are used for the calibration process. 


Technical details of these sub-systems are described below. 

\subsection{Antenna}
\label{sec:antenna}
[Nelis]

provide some text - not giving away too much of the second paper. 

Focus on hardware here, present the theoretical beam patterns that will be generated.  

Theory for the beam forming will be presented in another paper. 

\subsection{Analog Section}
\label{sec:analog}
[Manuel or Joe]

\subsection{Digital Section}
\label{sec:digital}

[Manuel or Joe ?]

\section{Measurement Theory}
 \label{section:Algorithm}

%
% JLG updates 3/6/2017
%
 The theory supporting the instrument conceptual design from section \ref{sec:instrument} is presented here. 
 First, the signal model at the input to the two digital receivers will be derived.
 This is followed by an expression for the general autocorrelation of each of these signals and the cross-correlation between them at multiple lags.  
 One important consideration is isolation between the direct and reflected signals.  
 In GNSS-R and other reflectometry measurements using comparably wide-band transmissions, the short correlation length of the transmitted signal allows the use of cross-correlation for isolating the direct and reflected signals through their differences in delay. 
 Alternatively, the direct and reflected signals can be isolated through use of high-gain, narrow beam antennas [Ku band altimetry, P-band snow]. 
 The P-band signals of opportunity, however, have narrow (25 KHz) bandwidths. 
 Complete isolation between the direct and reflected signals will require a difference in path length of at least 24~km. 
 The maximum altitude of the B-200 aircraft is only Attached is the latest outline of the proposal.

 
 
\subsection{Signal Model}
The transmitted signal from a satellite ($x_{T}$) is modeled as
\begin{equation}
	x_T(t)=\sqrt{C_T}a(t)e^{j\omega_et},
    \label{Eq:xT}
\end{equation}
where the transmitted carrier power is $C_{T}$,  and $\omega_{e}$ is the free space frequency. $a(t)$ is assumed the baseband data signal. $T_c$  is the chipping time, and $p$ is an uniform distribution  on integers 0, 1, 2, 3, and rect denotes a rectangle function. The received signals at the object with antenna along the direct and reflected path ray can be represented 
\begin{eqnarray}
	%\begin{split}
  	x_D(t)&=\sqrt{C_D}a(t-\tau_D)e^{j\omega_e(t-\tau_D)}, \\
    x_R(t)&=\sqrt{C_R}a(t-\tau_R)e^{j\omega_e(t-\tau_R)}, 
   % \end{split}
    \label{Eq:xD_xR}
\end{eqnarray}
where $C_D$ and $C_R$ are received powers of signals the direct and reflected path at the antennas.  
$\tau_D$ and $\tau_R$ are the path delay along the direct and reflected path. The reflectivity $\Gamma$ is defined as
\begin{equation}
	\Gamma=\frac{C_{RS}}{C_{DS}}
    \label{Eq:reflect}
\end{equation}
Where $C_{DS}$ and $C_{RS}$ are, respectively, the incident (direct) and reflected signal powers at the surface. 
For an airbonre recevier, we will neglect the additional path loss from the aircraft to the surface, and approximate $C_{DS}\approx C_D$ and $C_{RS}\approx C_R$. 
\begin{figure}[t!]
	\centering
    \centering
	\includegraphics[width=5 in]{pdf/Picture1.jpg}
	\caption{The gain of sky-view and earth-view antenna for the direct and reflected path rays..}
	\label{fig:Hardware}
\end{figure}
As shown in Figure \ref{fig:Hardware}, at the  receiver receives the signals, the models of signal for channel 1 and 2 are given by
\begin{align}
 	x_1(t)&= \sqrt{G_1} \left[ 
       \sqrt{G_{S,D} C_D} a(t-\tau_{D}) e^{-\omega_{IF} \tau_D} + 
       \sqrt{G_{S,R} C_R} a(t-\tau_{R}) e^{-\omega_{IF} \tau_R} 
      \right] + n_1(t)  \label{Eq: x1_model} \\ 
    x_2(t)&=\sqrt{G_2}\left[ 
         \sqrt{G_{E,D} C_D}  a(t-\tau_{D}) e^{-\omega_{IF} \tau_D} + 
         \sqrt{G_{E,R} C_R} a(t-\tau_{R}) e^{-\omega_{IF} \tau_R}\right] + n_2(t) 
  \label{Eq: x2_model}
\end{align}
$G_1$ and $G_2$ are the gain of channel 1 and 2. $G_{S,D}$ and $G_{S,R}$ are the gain of sky-view antenna gain along the direct and reflected path ray, and  $G_{E,D}$ and $G_{E,R}$ are the gain of earth-view antenna gain along the direct and reflected path ray. $n_1$ and $n_2$ are the thermal noise,  zero-mean white Gaussian noise with power-spectral density $\sigma^2_1$ and $\sigma^2_2$, for channel 1 and 2, respectively.  $\omega_{IF} = \omega_c - \omega_{LO}$ is the intermediate frequency, following down-conversion with a local oscillator frequency of $\omega_{LO}$. 

\subsection{Correlation}
The digital receiver can perform a short-time correlation between two signals ($x_p$ and $x_q$) at delay $\tau_s$, using an integration time $T_I$. 
\begin{equation}
	R_{pq}(\tau)  %=\langle x_p^*(t)x_q(t+\tau)\rangle
    =\frac{1}{T_I} \int_{T_I}x_p(t) x_q^*(t+\tau)dt 
   % R_{pq}(\tau-\tau_{qp}) =\langle x_p^*(t-\tau_p)x_q(t-\tau_q +\tau)\rangle
    \label{Eq: correlation_def}
\end{equation}
We will assume $T_I$ is sufficiently long and that modulation is generated from an infinitely-long, constant envelope digital signal,so that the autocorrlation of $a(t)$ \begin{equation}
R_a (\tau) = \frac{1}{T_I} \int_{T_I} a(t) a^*(t+\tau) dt
\end{equation}
is known and has the following properties.
\begin{align}
   \label{eqn:acfa1}
	R_{a}(0) &= 1   \\
    R_{a}(\tau) & = R_a^* (-\tau)
    \label{eqn:acfa2}
\end{align}
In practice, $R_a(\tau)$ can be computed from a model, if the modulation design is known, or it can be measured experimentally through observation of the direct line-of-sight signal $x_D(t)$ with pair of directional antennas.  The latter procedure is described in Appendix \ref{App: Ra} and was applied to this problem.  The receiver noise can be expressed as a single equivalent thermal noise source at the input of the receiver, with the following properties.
\begin{equation}
\begin{split}
	R_a(\tau)&=\frac{1}{T_I} \int_{T_I}a(t) a^*(t+\tau)dt  \\
    R_a(\tau-\tau_{RD})&=\frac{1}{T_I} \int_{T_I}a(t-\tau_D) a^*(t-\tau_R+\tau)dt  \\
    R_a(\tau+\tau_{RD})&=\frac{1}{T_I} \int_{T_I}a(t-\tau_R) a^*(t-\tau_D+\tau)dt 
 \end{split}
\end{equation}

The correlation of noise $n_1$ and $n_2$ are defined as
\begin{equation}
\begin{split}
    \frac{1}{T_I} \int_{T_I}n_1(t) n_1^*(t+\tau)dt  &=  \sigma_1^2 (R_n(\tau) + n_{RES})\\
	\frac{1}{T_I} \int_{T_I}n_2(t) n_2^*(t+\tau)dt  &=  \sigma_2^2 (R_n(\tau) + n_{RES})\\
    \frac{1}{T_I} \int_{T_I}n_1(t) n_2^*(t+\tau)dt  &=  \sigma_1 \sigma_2  n_{RES}
 \end{split}
\end{equation}
$R_n(\tau)$ is the auto-correlation of infinite random variables with Gaussian distribution, and mean and standard deviation are 0 and 1, respectively. $n_{RES}$ represents the difference between the autocorrelation with finite and infinite random variables.
	
\[ R_n(\tau) = \left\{ 
  \begin{array}{l l}
    1 & \quad \text{if $\tau=0$}\\
    0 & \quad \text{else}
  \end{array} \right.\]


 The auto-correlation $R_{11}$of $x_1$, auto-correlation $R_{22}$ of $x_2$, and cross-correlation $R_{12}$ of $x_1$ and $x_2$ can be derived by substituting (\ref{Eq: x1_model}) and (\ref{Eq: x2_model}) into (\ref{Eq: correlation_def}). After digitizing the signal ($\tau_s = \tau / f_s$), the expressions of these three correlation functions are as follows. 
\begin{equation}
    \begin{split}
	R_{11}(\tau_s) =  & \left\{  
      \left( g^2_{1SD} + g^2_{1SR} \right) R_a(\tau_s) + 
      \right. \\
       & g_{1SD} g_{1SR} 
     \left[ 
          R_a(\tau_s-\tau_{RD})  e^{j\omega_{IF}\tau_{RD}}+ 
          R_a(\tau_s+\tau_{RD}) e^{-j\omega_{IF} \tau_{RD} }
     \right] \\
      &+  \left. G_1\sigma^2_1 R_n(\tau_s)   \right\} 
      + n_{RES}(\tau_s)    
   \label{Eq:R11}
   \end{split}
\end{equation}

\begin{equation}
    \begin{split}
	R_{22}(\tau_s) =  & \left\{  
      \left( g^2_{2ED} + g^2_{2ER} \right) R_a(\tau_s) + 
      \right. \\
       & g_{2ED} g_{2ER} 
     \left[ 
          R_a(\tau_s-\tau_{RD})  e^{j\omega_{IF}\tau_{RD}}+ 
          R_a(\tau_s+\tau_{RD}) e^{-j\omega_{IF} \tau_{RD} }
     \right] \\
      &+  \left. G_2\sigma^2_2 R_n(\tau_s) +   \right\} 
      + n_{RES}(\tau_s)    
   \label{Eq:R22}
   \end{split}
\end{equation}

\begin{equation}
    \begin{split}
    	R_{12}(\tau_s) = & \left\{  
       \left( g_{1SD} g_{2ED} + g_{1SR}g_{2ER} \right) R_a(\tau_s) +
       \right. \\
       & \left. 
       g_{1SD} g_{2ER} R_a(\tau_s-\tau_{RD})  e^{j\omega_{IF} \tau_{RD}}+ 
       g_{1SR} g_{2ED} R_a(\tau_s+\tau_{RD})  e^{-j\omega_{IF} \tau_{RD}}
       \right\} + n_{RES}(\tau_s)    
   \label{Eq:R12}
   \end{split}
\end{equation}

The following substitutions have been made
\begin{eqnarray}
	g_{1SD} &= \sqrt{G_1 G_{S,D} C_D} \label{Eq:g1sd} \\
    g_{1SR} &= \sqrt{G_1 G_{S,R} C_R} \label{Eq:g1sr} \\
    g_{2ED} &= \sqrt{G_2 G_{E,D} C_D} \label{Eq:g2ed} \\
    g_{2ER} &= \sqrt{G_2 G_{E,R} C_R} \label{Eq:g2er}
\end{eqnarray}



\subsection{Definition of a Reflectivity Observable}
 \label{sec: Observable and Forward Model}
 
 [Zenki - update to latest, I will revise after Weds]
 
In order to derive the relationship between the numerical values of the three complex correlation functions $R_{11}$, $R_{22}$ and $R_{12}$, derived in the previous section, and the reflectivity, $\Gamma$, we will first consider an ideal case using  
antennas has perfect isolation of the desired signal, such that
 $G_{S,R}\approx 0 $ and $G_{E,D}\approx 0$. 
Moreover, assume a perfect downconversion to baseband, so that the immediate frequency $\omega_{IF} \approx 0$. 
A calibration process, described in section \ref{sec:calibration},  is used to obtain the antenna gains, $G_{S,D}$, $G_{S,R}$, $G_{E,R}$, $G_{E,D}$, the ratio of receiver gains, $(G_1/G_2)$ and the noise floors $\sigma_1$, $\sigma_2$. 
Once these parameters, assumed to remain constant after calibration, are obtained, the following observable is formed. 
\begin{eqnarray}
 \Gamma_{12} =	\sqrt{\frac{G_1}{G_2}}\sqrt{\frac{G_{S,D}}{G_{E,R}}} \frac{R_{12}(\tau^s_{RD})}{R_{11}(0)-G_1\sigma_1^2}  
    \label{Eq:Gamma_estimation_approx}
\end{eqnarray}
Substitution of (\ref{Eq:R11}), (\ref{Eq:R22}), and (\ref{Eq:R12}) into
(\ref{Eq:Gamma_estimation_approx}), with the assumptions that $G_{S,R}=0$, 
$G_{E,D}=0$ and $R_a(0)=1$, results in
\begin{equation}
\Gamma_{12} = \sqrt{\frac{G_1}{G_2}} \sqrt{\frac{G_{S,D}}{G_{E,R}}} 
  \frac{e^{-j \omega_c \tau_{RD}} g_{1SD} g_{2ER} +\sigma_1 \sigma_2 n_{RES}}
     {g_{1SD}^2 + \sigma_1^2 n_{RES}}
     \label{Eq:Gamma_estimation_approx_subst1}
\end{equation}
%\approx\sqrt{\hat{\Gamma}}e^{j\omega_e \tau_{RD}} 
Assuming a high signal to noise ratio, as would be expected for satellite communication signals, $n_{RES}$ can be combined into a single additive noise term using a first-order Taylor series expansion
\begin{equation}
    \frac{A+\sigma_1 \sigma_2 n_{RES}}{B+\sigma_1^2 n_{RES}} \approx \frac{A}{B} + 
   \frac{1}{B}\left[\sigma_1 \sigma_2 n_{RES}- \frac{A}{B}\sigma_1^2 n_{RES}\right] =\frac{A}{B}    +n_{\Gamma}
\end{equation}
if $|A|>> |\sigma_1 \sigma_2 n_{RES}|$ and $|B| >> |\sigma_1 ^2 n_{RES}|$.
Substituting (\ref{Eq:g1sd}) and (\ref{Eq:g2er}), with the high SNR assumption, reduces (\ref{Eq:Gamma_estimation_approx_subst1}) to 
\begin{equation}
    \Gamma_{12} = \sqrt{\Gamma} e^{-j \omega_c \tau_{RD}} + n_\Gamma
\end{equation}
in which $\tilde{n}_\Gamma$ is a zero-mean random variable. This shows one approach to estimating the reflectivity, $\Gamma$ is through forming the observable $\Gamma_{12}$ defined in (\ref{Eq:Gamma_estimation_approx}) and then computing the magnitude.  
Knowledge of the path delay, $\tau_{RD}$, is not required, as this will be manifested as a rotation of $\Gamma_{12}$ in the complex plane. 
A value for $\tau_{RD}$ is necessary for the argument of $R_{12}(\tau_{RD})$, however the required precision of this is not high, as the bandwidth of P-band SoOp signals is very low (25~KHz) which means that $R_a(\tau)$ will have a support of about 12~km in delay.  
The dependence of the phase of $\Gamma$ on the path delay will be much more sensitive, however, going through one complete cycle for a 1.2~m change in altitude.  
Path delay will depend upon the aircraft altitude and the terrain elevation. 
Through this formulation in the complex plane, the retrieval is not sensitive to accurate knowledge of the path delay.  This sensitivity will be explored later in section \ref{sec:error}.
Given the long wavelength of P-band signals and the practical limitations of designing an antenna for use on aircraft, an antenna with perfect isolation of the direct and reflected signals will not possible. 
An observable and forward model must be developed, which accounts for the interference between the direct and reflected signals, due to non-negligible gains $G_{S,R}$ and $G_{E,D}$.  In general, the $\Gamma_{12}$ observable can be written in the form of an observation equation as a function of two variable,s $\Gamma$ and the phase difference along the reflected ray path, $\phi = \omega_c \tau_{RD}$, with  additive noise.  $n_{\Gamma}$. 
\begin{equation}
\Gamma_{12} = f(\Gamma, \phi) +n_{\Gamma}
  \label{eqn:observation_eqn_general}
\end{equation}
A full expression for $f(\Gamma, \phi)$, removing the assumption of perfect isolation, is given by
\begin{equation}
f(\Gamma, \phi) =
	\frac{(\sqrt{I_E}+\sqrt{I_S}\Gamma)R_a(\tau_{RD})+\sqrt{\Gamma} 
    e^{j\phi}+\sqrt{I_S I_E\Gamma} R_a(2\tau_{RD})e^{-j\phi}} 
    {(1 + I_S \Gamma) + 2 \sqrt{I_S \Gamma} \mathcal{Re}[ R_a(\tau_{RD})] \cos \phi}
        \label{Eq:Gamma_estimation}
\end{equation}
%
%                   {(1 + I_S)\Gamma)R_a(0)+2\sqrt{I_S\Gamma} R_a(\tau^s_{RD})cos\phi}  
The isolation of sky-view and earth-view antenna are denoted as
\begin{eqnarray}
    I_S& = \frac{G_{SR}}{G_{SD}}\\
   I_E& = \frac{G_{ED}}{G_{ER}}
\end{eqnarray}

\subsection{Reflectivity Retrieval}  \label{sec: Reflectivity Retrieval}

[Zenki - update with latest, I will revise after Weds]
Equation (\ref{Eq:Gamma_estimation}), when plotted in the complex plane in Fig \ref{fig:Gamma12} will shows the effect of the antenna isolation, using the preliminary system design parameters from table \ref{tab:sysparam}. 
\begin{figure}[t!]
	\centering
    \centering
	\includegraphics[width=5 in]{pdf/Gamma12.jpg}
	\caption{The complex plot of $\Gamma_{12}$ with and without ideal antenna isolation; the black dot lines represent $\Gamma_{12}$ with ideal antenna isolation, the blue lines are $\Gamma_{12}$ with ideal antenna isolation, and reflectivities are 0.05, 0.15, 0.25, 0.35, 0.45, 0.55, and 0.65 from inner to outer contours.}
	\label{fig:Gamma12}
\end{figure}

In this figure, curves for the complex function $f(\Gamma, \phi) $are a set of reflectivities, over the range $-\pi \leq \phi \leq \pi$. On the top is the example of perfect isolation presented earlier, in which each reflectivity corresponds to a circle centered on the origin. As can be seen, the effects of non-zero $I_E$ and $I_S$ are to transform this circle into an ellipse, and shift its center away from the origin. The important consideration, however, is that the curves for different reflectivities do not intersect and thus there is a unique mapping between the real and imaginary parts of the observable and model.  Retrieval of a reflectivity estimate, $\hat{\Gamma}$ can therefore be found from the complex observable $\Gamma_{12}$ simply by numerical solution of the system of equations
\begin{equation}
\Gamma_{1,2} - f(\hat{\Gamma}, \hat{\phi}) = 0
\label{eqn:2x2solution}
\end{equation}
with $\hat{\phi}$ produced as an auxiliary  variable to account for the rotation in the complex plane due to the unknown path delay.  As with the ideal case, this eliminates the need for high-precision knowledge of the aircraft altitude or terrain.  Newton's method worked well for the solution of (\ref{eqn:2x2solution}). 

\begin{table}[ht]
\centering
\caption{System design specifications}
\begin{tabular}  {|c|c|c|c|}
	\hline
     \textbf{Parameter} & \textbf{Symbol}	& \textbf{Value} &	\textbf{units}  \\
     \hline
     Incident Direct Power &	$C_T$ & 14  & 	dBW	\\
          \hline
     Noise Power of channe 1&	 $\sigma_1^2$ & -131.0964  & 	dBW	\\
     \hline
      Noise Power of channel 2&	$\sigma_2^2$ & -130.2362  & 	dBW	\\
     \hline
     Gain of sky-view antenna along direct path ray &	$G_{S,D}$ & 6.7972  & 	dB	\\
    \hline
         Gain of earth-view antenna along reflected path ray&	$G_{SR}$ & 0.5123  & 	dB	\\
    \hline
         Gain of sky-view antenna along direct path ray &	$G_{ED}$ & 0.1192  & 	dB	\\
    \hline
             Gain of earth-view antenna along reflected path ray &	$G_{ER}$ & 4.9115  & 	dB	\\
    \hline
    Gain of channel 1 &	$G_1$ & 0  & 	dB	\\
    \hline
    Gain of channel 2 &	$G_2$ & 0  & 	dB	\\
    \hline
\end{tabular}
\label{tab:sysparam}
\end{table}

\section{Calibration}
[Zenki]
\label{sec:calibration}
In the reflectivity estimation described above, there are six parameters which must calibrated for the system, antenna gains in the directions of direct and reflected ray paths ($G_{S,D}$, $G_{S,R}$, $G_{E,D}$, $G_{E,R}$, the channel gain ratio ($G_1/G_2$), and the two noise floors $\sigma_1$, $\sigma_2$. 
Of these, we will assume that the four antenna gain values are properties of the antenna design and integration on the airframe and will remain constant. 
A companion paper, describing the antenna design and the in-flight calibration process, is being written.  
For the present study, we will use nominal values for these gains, as given in table \ref{tab:sysparam}.
The noise power and receiver gains, however, are expected to vary with time and will need to be periodically calibrated. 
Two approaches are used for this calibration, antenna swapping and injection of a noise source. 
Figure \ref{fig:Hardware} shows the architecture of calibration chain.

\bf I did not edit the subsections on the calibration - only broke a longer section into the 2 subsections \rm

\subsubsection{Calibration Noise Load}
In Figure \ref{fig:Hardware}, $T_{cal}$ is noise temperature of the calibrated noise load, $T_{REF}$ is noise temperature of RF terminator, $T_{sys1,pre}$ and $T_{sys2,pre}$is noise of channel 1 and 2, respectively, prior the transfer switch, $T_{sys1,post}$ and $T_{sys2,post}$is noise of channel 1 and 2, respectively, after the transfer switch, and $T_{ant,S}$ and $T_{ant,E}$is noise of sky-view and earth-view antenna, respectively. There are three modes of calibration chain. The noise power $\sigma^2$ is given by
\begin{equation}
	\sigma^2=\kappa TB
\end{equation}
Where $\kappa$ is the Boltzmen constant, $T$ is noise temperature, and $B$ is bandwidth. The first mode is normal mode which means there is no additional noise in the system, and the correlations $\tilde{R}^n$ in Equation \ref{Eq: R11_reduce}, \ref{Eq: R22_reduce}, and \ref{Eq: R11_reduce} can be rewrote as
\begin{eqnarray}
	\tilde{R}_{11}^n(\tau^s) &&= g^2_{1SD} R_a(\tau^s)e^{j\Delta\omega_e\tau_s}+
G_1(\sigma^2_{Ant,S}+\sigma^2_{sys1,pre} +\sigma^2_{sys1,post}) \delta(\tau^s)                                      
\label{Eq: R11_reduce_norm} \\
	\tilde{R}_{22}^n(\tau^s) &&= g^2_{2ER} R_a(\tau^s)e^{j\Delta\omega_e\tau_s}+
G_2(\sigma^2_{Ant,E}+\sigma^2_{sys2,pre} +\sigma^2_{sys2,post}) \delta(\tau^s)                                       
\label{Eq: R22_reduce_norm} \\
	\tilde{R}_{12}^n(\tau^s) &&= g_{1SD} g_{2ER} R_a(\tau^s-\tau^s_{RD})e^{j\omega_e \tau_{RD}} e^{j\Delta\omega_e\tau_s}  
\label{Eq: R12_reduce_norm}
\end{eqnarray}
Once the transfer switch is activated, the the correlations $\tilde{R}^s$ become
\begin{eqnarray}
\tilde{R}_{11}^s(\tau^s) &&= g^2_{1ER} R_a(\tau^s)e^{j\Delta\omega_e\tau_s}+
G_1(\sigma^2_{Ant,E}+\sigma^2_{sys2,pre} +\sigma^2_{sys1,post}) \delta(\tau^s)                                       
\label{Eq: R22_reduce_swap} \\
\tilde{R}_{22}^s(\tau^s) &&= g^2_{2SD} R_a(\tau^s)e^{j\Delta\omega_e\tau_s}+
G_2(\sigma^2_{Ant,S}+\sigma^2_{sys1,pre} +\sigma^2_{sys2,post}) \delta(\tau^s)                                      
\label{Eq: R11_reduce_swap} \\
	\tilde{R}_{12}^s(\tau^s) &&= g_{1SD} g_{2ER} R_a(\tau^s+\tau^s_{RD})e^{-j\omega_e \tau_{RD}} e^{j\Delta\omega_e\tau_s}  
\label{Eq: R12_reduce_swap}
\end{eqnarray}
When the system injects noise from noise load, the the correlations $\tilde{R}^c$ is changed as
\begin{eqnarray}
	\tilde{R}_{11}^c(\tau^s) &&= g^2_{1SD} R_a(\tau^s)e^{j\Delta\omega_e\tau_s}+
G_1(\sigma^2_{Ant,S}+\sigma^2_{sys1,pre} +\sigma^2_{sys1,post}+\sigma^2_{cal}) \delta(\tau^s)                                      
\label{Eq: R11_reduce_cal} \\
	\tilde{R}_{22}^n(\tau^s) &&= g^2_{2ER} R_a(\tau^s)e^{j\Delta\omega_e\tau_s}+
G_2(\sigma^2_{Ant,E}+\sigma^2_{sys2,pre} +\sigma^2_{sys2,post}+\sigma^2_{cal}) \delta(\tau^s)                                       
\label{Eq: R22_reduce_cal} \\
	\tilde{R}_{12}^n(\tau^s) &&= g_{1SD} g_{2ER} R_a(\tau^s-\tau^s_{RD})e^{j\omega_e \tau_{RD}} e^{j\Delta\omega_e\tau_s}\delta(\tau^s)+
    \sqrt{G_1 G_2}\sigma^2_{cal}\delta(\tau^s)  
\label{Eq: R12_reduce_cal}
\end{eqnarray}The last mode switches the signal sources to reference load in the following equation.
\begin{eqnarray}
	\tilde{R}_{11}^r(\tau^s) &&= G_1(\sigma^2_{REF}+\sigma^2_{sys1,post}) \delta(\tau^s)                                      
\label{Eq: R11_reduce_ref} \\
	\tilde{R}_{22}^r(\tau^s) &&= G_2(\sigma^2_{REF}+\sigma^2_{sys2,post}) \delta(\tau^s)                                
\label{Eq: R22_reduce_ref} \\
	\tilde{R}_{12}^r(\tau^s) &&= 0 
\label{Eq: R12_reduce_ref}
\end{eqnarray}
\section{The injection of noise from noise source and terminator}
The channel gains $G_1$ and $G_2$ are estimated from  $\tilde{R}^n_*$ in Equations \ref{Eq: R11_reduce_norm}-\ref{Eq: R12_reduce_norm} and $\tilde{R}^c_*$ in Equations \ref{Eq: R11_reduce_cal}-\ref{Eq: R12_reduce_cal}, and the noise power from noise load $\sigma^2_{cal}$. The product of channel gain and $\sigma^2_{cal}$ is calculated in
\begin{eqnarray}
	\tilde{R}^c_{11}(0) - \tilde{R}^n_{11}(0)&&= G_1\sigma^2_{cal} \label{Eq: Cal_channel_gain1a} \\
    \tilde{R}^c_{22}(0) - \tilde{R}^n_{22}(0)&&= G_2\sigma^2_{cal} \label{Eq: Cal_channel_gain2a} \\
    \tilde{R}^c_{12}(\tau_{RD}) - \tilde{R}^n_{12}(\tau_{RD})&&= \sqrt{G_1 G_2}\sigma^2_{cal} \label{Eq: Cal_channel_gain12a}
\end{eqnarray}
Once $\sigma^2_{cal}$ is well measured, 
\begin{eqnarray}
	G_1 &&= \frac{\tilde{R}^c_{11}(0) - \tilde{R}^n_{11}(0)}{\sigma^2_{cal}}  \label{Eq: Cal_channel_gain1} \\
    G_2 &&= \frac{\tilde{R}^c_{22}(0) - \tilde{R}^n_{22}(0)}{\sigma^2_{cal}}  \label{Eq: Cal_channel_gain2} \\
    \sqrt{G_1 G_2} &&= \frac{\tilde{R}^c_{12}(\tau_{RD}) - \tilde{R}^n_{12}(\tau_{RD})}{\sigma^2_{cal}}  \label{Eq: Cal_channel_gain12}
\end{eqnarray}

The noise of channel 1 and 2 after the transfer switch are further estimated by substituting $G_1$ and $G_2$ into Equations \ref{Eq: R11_reduce_ref} and \ref{Eq: R22_reduce_ref}. The Equations \ref{Eq: R11_reduce_ref} and \ref{Eq: R22_reduce_ref} are rewrote as
\begin{eqnarray}
	\sigma^2_{sys1,post} &&= \frac{\tilde{R}^r_{11}(0) }{G_1}-\sigma^2_{REF}  \label{Eq: Cal_channel_noise1} \\
    \sigma^2_{sys2,post} &&= \frac{\tilde{R}^r_{22}(0) }{G_2}-\sigma^2_{REF}  \label{Eq: Cal_channel_noise2} 
\end{eqnarray}

\subsubsection{Antenna swapping}
The antenna swapping \cite{Alejandro:2013} is used for the estimation of the ratio of channel gain $\frac{G_1}{G_2}$, and it can be measured by the ratio of $\tilde{R}^s$ Equations \ref{Eq: R11_reduce_swap}-\ref{Eq: R12_reduce_swap} and $\tilde{R}^n$Equations \ref{Eq: R11_reduce_norm}-\ref{Eq: R12_reduce_norm}. When the noise of channels are approximately same, the ratio of channel gain is calculated as
\begin{eqnarray}
	\frac{G_1}{G_2} = \frac{\tilde{R}^n_{11}}{\tilde{R}^s_{11}} \\
    \frac{G_1}{G_2} = \frac{\tilde{R}^s_{22}}{\tilde{R}^n_{22}}  \label{Eq: Cal_channel_gain_ratio}
\end{eqnarray}


\bf The following sections are outlines, with bullet points.  I have moved some of the UFO -system specific description into one of the sections, to describe the experimetnal work \rm

%JLG\section{Measurement Simulation}

To verify the measurement theory, we establish a comprehensive simulator from  satellites to receiver with Matlab. The simulator includes signal resource, path ray of transmitted signal, antenna gain, and receiver parameters.

\subsection{Signal source}
The band

%JLG\section{Uncertainty analysis}

Parameters of uncertainty analysis inlude antenna gains ($G_{S,D}, G_{S,R}, G_{E,D}, G_{E,R}$), channel gain ($G_1, G_2$), noise power spectra ($\sigma_1$), reflectivity ($\Gamma$), and phase ($\phi$).  First, (\ref{eqn:observation_eqn_general}) without noise residual  term is expressed as
\begin{equation}
g(x, \tilde{x}) = \Gamma_{12}(x, \tilde{x}) - f(\tilde{x})
\end{equation}
Where $x=\{G_{S,D}, G_{S,R}, G_{E,D}, G_{E,R}, G_1, G_2, \sigma_1, \Gamma, \phi\}$ and $\tilde{x}=\{\tilde{G}_{S,D}, \tilde{G}_{S,R}, \tilde{G}_{E,D}, \tilde{G}_{E,R}, \tilde{G}_1, \tilde{G}_2, \tilde{\sigma}_1, \tilde{\Gamma}, \tilde{\phi}\}$ are the set of the parameters for the measurements and calibration terms. Once the claibration terms $\tilde{x}$ exactly matches $x$, $g(\Gamma, \phi, x, \tilde{x}) = 0$. The full expression of $g(\Gamma,\phi)$ is
\begin{equation}
g(x,\tilde{x}) = \sqrt{\frac{\tilde{G}_1}{\tilde{G}_2}} \sqrt{\frac{\tilde{G}_{S,D}}{\tilde{G}_{E,R}}}  
\frac{R_{1,2}(\tau^s_{RD})}{R_{1,1}(0)-\tilde{G}_1 \sigma^2_1} -
 \frac
{\sqrt{\tilde{I}_E}+ \sqrt{\tilde{I}_S} \tilde{\Gamma})R_a(\tau_{RD})+
\sqrt{\tilde{\Gamma}}R_a(0)e^{j\tilde{\phi}}+
\sqrt{\tilde{I}_E \tilde{I}_S\tilde{\Gamma}}R_a(2\tau_{RD})e^{-j \tilde{\phi} } )
}
{(1+\tilde{I}_S \tilde{\Gamma})R_a(0)+
2 \sqrt{\tilde{I}_S\tilde{\Gamma}}(\mathcal{Re} (R_a(\tau_{RD})) )cos\tilde{\phi}
}                            
\end{equation}
The total derivative of $g(x,\tilde{x})$ is
\begin{align}
dg(x,\tilde{x})&=\sum_{x_i \in \tilde{x}} \frac{\partial(g)}{\partial(x_i)} dx_i \\
\Rightarrow dg(x,\tilde{x})&=\sum_{x_i \in \tilde{x}} (\frac{\partial(\Gamma_{12})}{\partial(x_i)} dx_i + \frac{\partial(f)}{\partial(x_i)}dx_i)
\end{align}
The first term is the total derivative of $\Gamma_{12}$, and the following equations shows the partial derivative of different parameters.
\begin{equation}
\tilde{G}_1 :
\sqrt{\frac{\tilde{G}_2}{\tilde{G}_1}}
\sqrt{\frac{\tilde{G}_{S,D}}{\tilde{G}_{E,R}}}
\frac{R_{1,2}(\tau_{RD}) (R_{1,1}(0) + \tilde{G}_1 \sigma_1^2)}
{2\tilde{G}_2(R_{1,1}(0) - \tilde{G}_1 \sigma_1^2)^2 )} 
\end{equation}
\begin{equation}
\tilde{G}_2 :
\sqrt{\frac{\tilde{G}_2}{\tilde{G}_1}}
\sqrt{\frac{\tilde{G}_{S,D}}{\tilde{G}_{E,R}}}
-\frac{R_{1,2}(\tau_{RD})}
{2\tilde{G}_2^2(R_{1,1}(0) - \tilde{G}_1 \sigma_1^2) )} 
\end{equation}
\begin{equation}
\tilde{G}_{S,D}:
 \sqrt{\frac{\tilde{G}_1}{\tilde{G}_2}}
 \sqrt{\frac{\tilde{G}_{E,R}}{\tilde{G}_{S,D}}}  
\frac{R_{1,2}(\tau^s_{RD})}{2 G_{E,R} (R_{1,1}(0)-\tilde{G}_1 \sigma^2_1)} 
\end{equation}
\begin{equation}
\tilde{G}_{S,D}:
 -\sqrt{\frac{\tilde{G}_1}{\tilde{G}_2}}
 \sqrt{\frac{\tilde{G}_{E,R}}{\tilde{G}_{S,D}}}  
\frac{R_{1,2}(\tau^s_{RD})}{2 G_{E,R}^2 (R_{1,1}(0)-\tilde{G}_1 \sigma^2_1)} 
\end{equation}\begin{equation}
\tilde{\sigma}_1:
\sqrt{\frac{\tilde{G}_1}{\tilde{G}_2}} \sqrt{\frac{\tilde{G}_{S,D}}{\tilde{G}_{E,R}}}  
\frac{2 \tilde{G}_1 \tilde{\sigma}_1 R_{1,2}(\tau^s_{RD})}{(R_{1,1}(0)-\tilde{G}_1 \sigma^2_1)^2} 
\end{equation}
   
\section{Sensitivity Analysis and Error Budget}

[Zenki]
\label{sec:error}
   \begin{itemize}
     \item Error sources: antenna gain calibration uncertainty, aircraft motion, noise calibration uncertainty, etc.
     \item Show sensitivity to error in each one of these and how it maps to error in the retrieval of reflectivity
     \item special case of aircraft motion and integration time.
     \item Summary table for each error source with an RSS of the error - this will be in the form of a budget - our requirements on each component of the system - to meet our working requirement.
     \item End-to-end "ensemble" simulation, of a large number of cases distributed over a range of soil moistures, noise figures, etc ...
   \end{itemize}
   
\subsection{Sensitivity Analysis}
The performance of reflectivity retrieval highly depends on the offset of calibrated terms $(G_1, G_2, G_{S,D}, G_{S,R}, G_{E,D}, G_{E,R}, \sigma_1)$ in Equation \ref{Eq:Gamma_estimation}. Equation \ref{Eq:Gamma_estimation} is rewritten as
\begin{equation}
f(\Gamma, \phi, G_1, G_2, G_{S,D}, G_{S,R}, G_{E,D}, G_{E,R}, \sigma_1) =
	\frac{(\sqrt{I_E}+\sqrt{I_S}\Gamma)R_a(\tau_{RD})+\sqrt{\Gamma} 
    e^{j\phi}+\sqrt{I_S I_E\Gamma} R_a(2\tau_{RD})e^{-j\phi}} 
    {1 + I_S \Gamma + 2 \sqrt{I_S \Gamma} \mathcal{Re}[ R_a(\tau_{RD})] \cos \phi}
        \label{Eq:Gamma_uncertainty}
\end{equation}

To analyze the effect of the calibrated terms,  
From Section \ref{sec: Observable and Forward Model},  includes  . . $\Gamma_{12}$ is an implicit function, and the relationship between $\Gamma$ and the calibrated terms is derived by

\begin{eqnarray}
d\Gamma_{12} = \frac{\partial \Gamma_{12}}{\partial G_1} dG_1 
\end{eqnarray}


\section{Conclusion}
The conclusion goes here.





% if have a single appendix:
%\appendix[Proof of the Zonklar Equations]
% or
%\appendix  % for no appendix heading
% do not use \section anymore after \appendix, only \section*
% is possibly needed

% use appendices with more than one appendix
% then use \section to start each appendix
% you must declare a \section before using any
% \subsection or using \label (\appendices by itself
% starts a section numbered zero.



\appendices

\section{P-band Signal Sources}
[Zenki]
\label{app:signal_source}
   \begin{itemize}
     \item General description of allocation of 230-270 MHz band for Govt use
     \item Published information about UFO - signal design, link budget, and orbits - be sure to cite the source of data
     \item Description of field experiment to verify the properties, including the ambiguity function, nominal values for signal power, persistence of signals, etc ... 
     \item summary of properties of UFO, and reference to table \ref{tab:sysparam}
\item Link budget
\item ambigiuty function experimentlaly verified
\item direction finding to validate ephemeris ???
   \end{itemize}
   
   \bf this is directly from Zenki's draft: \rm 
   
   The Ultra High Frequency Follow-On (UFO) system is a communication satellite system whose function is to provide communications for airborne, ship, submarine and ground forces. UFO satellite system is a DOD sponsored program and operated by U.S. Navy. By now there are totally 11 satellites. The first launch is in Mar.1993 and the planned lifetime of UFO system is 14 years. It replaced FLTSATCOM and is scheduled to be replaced by MUOS. The orbit of UFO is geosynchronous orbit and the altitude is 32810 km. From F-1 to F-7, UFO satellites provide redundant command and ranging capabilities when the satellite is on station. The bandwidth of each satellite is 25 kHz and it is on board processing. According to the amateur radio website \cite{Matt:2014}, the frequencies from both satellites are as the following table. Table\ref{Table Band Plan} indicates the location for band plan for UFO satellites.

\begin{table}[ht]
\centering
\begin{tabular}  {|c|c|c|c|c|}
	\hline
     \textbf{Band Plan} & \textbf{November}	& \textbf{Quebec} &	\textbf{Papa} & \textbf{Oscar} \\
    \hline
    Name & UFO F5 & UFO F6 & UFO F7 & UFO F2 \\
    \hline
     Location &	105.6° W & 99.2° W & 	22.0° W	&  28.7° E\\
    \hline
    Name & UFO F11 & UFO F10 & UFO F9  & UFO F3 \\
    \hline
    Location& 71.1° E &	72.6° E &	135.0° E &	156.4° W \\
    \hline
\end{tabular}
\caption{The location and band plan of UFO satellites}
\label{Table Band Plan}
\end{table}

\section{Correlation of modulation of signal} \label{App: Ra}
The modulation of signal is Assumed to be QPSK
\begin{equation}
a(t)=\sum_{q=0}^{\infty}rect(\frac{t-(q+1/2)T_c}{T_c})e^{j(2p(q)+1)\frac{\pi}{4}}
\end{equation}
[JLG - not sure this is needed - isnt this just standard comm theory ? ]

Autocorrelation of a discrete time signal $x[n]$ with finite samples $Nn_c$ is obtained directly by circular convolution or operating from the frequency domain in the following equations.
\begin{equation}\label{Eq: Auto_corx}
\begin{split}
R_x[n] &= \sum\limits_{m=0}^{Nn_c-1} x^*[m] x[n+m \mod Nn_c] \\
	 &= \frac{1}{Nn_c}\sum\limits_{k=0}^{Nn_c-1} X^*[k]X[k] e^{2\pi j \frac{kn}{Nn_c}},
\end{split}
\end{equation}
where $X[k]$ is $x[n]$ in frequency domain by Discrete Fourier Transform (DFT), and the relationship between $x[n]$ and $X[k]$ is expressed in the following equations.
\begin{equation}\label{Eq: xn_XK}
\begin{split}
X[k] &= \sum\limits_{m=0}^{Nn_c-1} x[n] e^{-2\pi j \frac{kn}{Nn_c}} \\
x[n] &= \frac{1}{Nn_c} \sum\limits_{k=0}^{Nn_c-1} X[k] e^{2\pi j \frac{kn}{Nn_c}} 
\end{split}
\end{equation}
Let $a[n]$ denote the discrete time QPSK signal with $Nn_c$ samples and be expressed in the form
\begin{equation}\label{Eq: a_n}
\begin{split}
a[n] &= \sum\limits_{q=0}^{N-1} rect(\frac{nT_s-(q+1/2)T_c}{T_c})e^{j(2\mathcal{U}_q+1)\frac{\pi}{4}} \\ 
	 &= \sum\limits_{q=0}^{N-1} rect(\frac{n-(q+1/2)n_c}{n_c})e^{j(2\mathcal{U}_q+1)\frac{\pi}{4}},    
     \end{split}
\end{equation}
\begin{figure}[t!]
	\centering
	\includegraphics[width=3 in]{pdf/a_n.jpg}
	\caption{The discrete time rectangular function}
	\label{fig:a_n}
\end{figure}
where $T_s$ is the sampling time, $T_c=1/bw$ is the time corresponding to bandwidth ($bw$), $n_c=T_c/T_s$ is the number samples within $T_c$, $N$ represents numbers of $n_c$ ,and $rect$ denotes a rectangular function as shown in Figure \ref{fig:a_n}. $\mathcal{U}_q$ is uniform distribution with integers 0, 1, 2, and 3. To elaborate the autocorrelation of $a[n]$, the autocorrelation of the single rectangular function $c[n]$ is derived first. 
\begin{equation}\label{Eq: c_n}
c[n] = rect(\frac{n-\frac{1}{2} n_c}{n_c})
\end{equation}
The autocorrelation of $c[n]$ is denoted by $d[n]$, and is derived by substituting Equation \ref{Eq: c_n} into Equation \ref{Eq: Auto_corx}
\begin{equation}\label{Eq: d_n}
\begin{split}
d[n] &= \sum\limits_{m=0}^{Nn_c-1} c^*[m] c[n+m] \\
	 &= \sum\limits_{m=0}^{Nn_c-1} rect(\frac{m-\frac{1}{2} n_c}{n_c}) rect(\frac{n+m-\frac{1}{2} n_c \mod Nn_c}{n_c}) \\
     &= \sum\limits_{m=0}^{n_c-1} (n_c-n)\delta[n-m] + 
        \sum\limits_{m=(N-1)n_c-1}^{Nn_c-1} (n-(N-1)n_c-2)\delta[n-m] 
\end{split}
\end{equation}
Figure \ref{fig:c_n_d_n} illustrates $c[n]$ and $d[n]$. The autocorrelation of $c[n]$ in frequency domain is also obtained by Equation \ref{Eq: Auto_corx}.
\begin{equation}\label{Eq: D_k}
\begin{split}
D[k] &= C^*[k] C[k] \\
	 &= (\sum\limits_{n=0}^{Nn_c-1} rect(\frac{n-\frac{1}{2} n_c}{n_c}) e^{-2\pi j \frac{kn}{Nn_c}})^* 
     (\sum\limits_{n=0}^{Nn_c-1} rect(\frac{n-\frac{1}{2} n_c}{n_c}) e^{-2\pi j \frac{kn}{Nn_c}} ) \\
     &= (\frac{ sin(\frac{\pi k}{N})}{sin(\frac{\pi k}{N n_c}}))^2
\end{split}
\end{equation}
\begin{figure}[t!]
	\centering
	\includegraphics[width=3 in]{pdf/rect_auto.jpg}
	\caption{The discrete time rectangular function}
	\label{fig:c_n_d_n}
\end{figure}
The autocorrelation of $a[n]$ denoted as $R_a[n]$ is derived in 
\begin{equation}\label{Eq: Ra_n}
R_a[n] = \frac{1}{Nn_c}\sum\limits_{k=0}^{Nn_c-1} A^*[k]A[k] e^{2\pi j \frac{kn}{Nn_c}}
\end{equation}
By DFT, $A[k]$ converted from $a[n]$ is given by
\begin{equation}\label{Eq: DFT_a_n}
A[k] = \sum\limits_{n=0}^{Nn_c} a[n]e^{-j2\pi \frac{kn}{Nn_c}}
\end{equation}
After substituting Equation \ref{Eq: a_n} into Equation \ref{Eq: DFT_a_n}, $A[k]$ is derived as
\begin{equation}\label{Eq: A_k}
\begin{split}
A[k] &= \sum\limits_{n=0}^{Nn_c} \sum\limits_{q=0}^{N-1} rect(\frac{n-(q+1/2)n_c}{n_c}) e^{j(2\mathcal{U}_q+1)\frac{\pi}{4}} e^{-j2\pi \frac{kn}{Nn_c}} \\ 
     &= \sum\limits_{q=0}^{N-1} \sum\limits_{n=0}^{Nn_c} rect(\frac{n-(q+1/2)n_c}{n_c}) e^{-j2\pi \frac{kn}{Nn_c}} e^{j(2\mathcal{U}_q+1)\frac{\pi}{4}} \\ 
     &= \sum\limits_{q=0}^{N-1} \sum\limits_{n=qn_c}^{(q+1)n_c} e^{-j2\pi \frac{kn}{Nn_c}} e^{j(2\mathcal{U}_q+1)\frac{\pi}{4}} \\
     &= \frac{1-e^{-j2\pi\frac{k}{N}}} {1-e^{-j2\pi\frac{k}{Nn_c}}} e^{-j2\pi \frac{qk}{N}} \sum\limits_{q=0}^{N-1} e^{j(2\mathcal{U}_q+1)\frac{\pi}{4}} \\
     &= \frac{sin(\pi\frac{k}{N})} {sin(\pi\frac{k}{Nn_c)}} e^{-j\pi \frac{k(n_c-1)}{Nn_c}} \sum\limits_{q=0}^{N-1} e^{-j2\pi \frac{qk}{N}} e^{j(2\mathcal{U}_q+1)\frac{\pi}{4}} 
\end{split}
\end{equation} 
$R_a[n]$ is further derived by substituting Equation \ref{Eq: A_k} into Equation \ref{Eq: Ra_n}.
\begin{equation}\label{Eq: IDFT_Ra1}
\begin{split}
R_a[n] &= \frac{1}{Nn_c}\sum\limits_{k=0}^{Nn_c} D[k] 
\sum\limits_{q=0}^{N-1} \sum\limits_{p=0}^{N-1} 
e^{j(\mathcal{U}_q-\mathcal{U}_p)\frac{\pi}{2}} 
e^{-j2\pi \frac{(q-p)k}{Nn_c}}
e^{j2\pi \frac{kn}{Nn_c}} \\
 &= \sum\limits_{q=0}^{N-1} \sum\limits_{p=0}^{N-1} 
e^{j(\mathcal{U}_q-\mathcal{U}_p)\frac{\pi}{2}} 
\frac{1}{Nn_c}\sum\limits_{k=0}^{Nn_c} D[k] e^{jc\pi \frac{(n-(q-p)n_c) k}{Nn_c}}
\end{split}
\end{equation}
According to Equation \ref{Eq: xn_XK}, $d[n]$ is converted from $D[k]$ by IDFT. After replacing $n$ to $n-(q-p)$, 
\begin{equation}\label{Eq: dn_tmp}
d[n-(q-p)n_c] = \frac{1}{Nn_c} \sum\limits_{k=0}^{Nn_c} D[k] e^{jc\pi \frac{(n-(q-p)n_c) k}{Nn_c}} 
\end{equation}
In the end, the autocorrelation of $a_n$ is simplified as in
\begin{equation}\label{Eq: IDFT_Ra2}
R_a[n] = \sum\limits_{q=0}^{N-1} \sum\limits_{p=0}^{N-1} 
e^{j(\mathcal{U}_q-\mathcal{U}_p)\frac{\pi}{2}} d[n-(q-p)n_c]
\end{equation}
When $q = p$, Equation \ref{Eq: IDFT_Ra2} becomes
\begin{equation}\label{Eq: IDFT_Ra_q_eq_p}
R_a[n] = N d[n]
\end{equation}
If $q \neq p$, The uniform difference distribution $\mathcal{D} = \mathcal{U}_q-\mathcal{U}_p$ of two uniform integers 0, 1, 2, and 3 is found as 
\begin{equation}\label{Eq: diff_uni}
P_{\mathcal{D}}[n] = \frac{4-|n|}{16}, n \in \{ -3,-2,-1,0,1,2,3 \} 
\end{equation}
Equation \ref{Eq: IDFT_Ra2} is rewritten as in
\begin{equation}\label{Eq: IDFT_Ra3}
R_a[n] = \sum\limits_{\substack{q=-(N-1) \\ q\neq 0}}^{N-1} e^{j(\mathcal{D}_q)\frac{\pi}{2}} d[n-q n_c]
\end{equation}
When $p \neq q $, the expected value of $R_a[n]$ is derived in the following equations
\begin{equation}\label{Eq: E_Ra}
\begin{split}
E[R_a[n]] &= \sum\limits_{p=-3}^{p=3} \sum\limits_{\substack{q=-(N-1) \\ q\neq 0}}^{N-1}  e^{j p\frac{\pi}{2}} d[n-q n_c] P_{\mathcal{D}}[p] \\
          &= \sum\limits_{\substack{q=-(N-1) \\ q\neq 0}}^{N-1} d[n-q n_c]  \sum\limits_{p=-3}^{p=3} e^{j p\frac{\pi}{2}}  P_{\mathcal{D}}[p] \\
          &= \sum\limits_{\substack{q=-(N-1) \\ q\neq 0}}^{N-1} d[n-q n_c] [(j) \frac{1}{16} + (-1) \frac{1}{8} + (-j) \frac{3}{16} + (1) \frac{1}{4} + (j) \frac{1}{16} + (1) \frac{1}{16} + (-j) \frac{1}{16}] \\
          &= 0
\end{split}
\end{equation}
By combining Equations \ref{Eq: IDFT_Ra_q_eq_p} and \ref{Eq: E_Ra}, the expected value of $R_a[n]$ is
\begin{equation}\label{Eq: E_Ra2}
E[R_a[n]] = N d[n]
\end{equation}
After subtituing Equation \ref{Eq: d_n} into Equation \ref{Eq: E_Ra2}, the expected value of $R_a[n]$ becomes
\begin{equation}\label{Eq: E_Ra3}
E[R_a[n]] = \sum\limits_{m=0}^{n_c-1} N(n_c-n)\delta[n-m] + 
            \sum\limits_{m=(N-1)n_c-1}^{Nn_c-1} N(n-(N-1)n_c-2)\delta[n-m] 
\end{equation}




\section{Working area- stuff cut out}

Most of this wont be needed - we will replace it with a citation to the relevant papers. I left a much reduced version in the first section of the paper. 

%\section{Soil moisture and reflectivity}
Dielectric constant of soil is a function Reflectivity is estimated by the measurement of SoOp, and the soil Dielectric constant is a very import factor for studying the reflectivity. The dielectric constant of media depends on the properties of media, for example, the dielectric constant of water can be calculated by temperature, salinity, and frequency.


Reflectivity $\Gamma$ is the fraction of incident electromagnetic power that is reflected at an interface. The reflectivity can be represented by Fresnel refection coefficients $\mathcal{R}$.

\begin{equation}
\Gamma = |\mathcal{R}|^2=\frac{P_D}{P_R}
\end{equation}

Where $P_D$ and $P_R$ are the power of direct signal and reflection, respectively. $\mathcal{R}_{hh}$ and $\mathcal{R}_{vv}$ are the co-polar refection coefficients of horizontal plane $(h)$ and vertical plane $(v)$, and they can be expressed as

\begin{equation} 
\begin{split}
	{\mathcal{R}_{hh}} = {}& \frac{{{{\tilde \varepsilon }_r}\cos \theta  - \sqrt 				{{{\tilde \varepsilon }_r} - {{\sin }^2}\theta } }}{{{{\tilde \varepsilon}_r}				\cos \theta  + \sqrt {{{\tilde \varepsilon }_r} - {{\sin }^2}\theta }}}
\\
	{\mathcal{R}_{vv}} = {}& \frac{{\cos \theta  - \sqrt {{{\tilde \varepsilon }_r} - {{\sin }^2}\theta } }}{{\cos \theta  + \sqrt {{{\tilde \varepsilon }_r} - {{\sin }^2}\theta }}} 
    \end{split}
     \label{Eq: reflectivity_di}
\end{equation}

Where $\tilde{\varepsilon}_r$ is the complex relative complex dielectric constant  $\theta$ is the incident angle of signal. The circular polarization cab be transformed by the linear polarization. The transition matrix $\mathcal{T}$ is

\begin{equation*}
	\begin{matrix}
	\hat{u}_{r} = \frac{1}{\sqrt{2}}(\hat{u}_h - j\cdot \hat{u}_v) \\
	\hat{u}_{l} = \frac{1}{\sqrt{2}}(\hat{u}_h + j\cdot \hat{u}_v)
	\end{matrix}
	\Rightarrow 
	\begin{bmatrix}
	\hat{u}_{r} \\
	\hat{u}_{l}
	\end{bmatrix}= \frac{1}{\sqrt{2}}	 
	\begin{bmatrix}
	1 & -j \\
	1 & j
	\end{bmatrix}	
	\begin{bmatrix}
	\hat{u}_{h} \\
	\hat{u}_{v}
	\end{bmatrix}
\end{equation*}

\begin{equation}
	\mathcal{T}_{rl \rightarrow hv} =  \frac{1}{\sqrt{2}}	
	\begin{bmatrix}
	1 & -j \\
	1 & j
	\end{bmatrix}
\end{equation}

The coordinate of co-plane and cross plane for linear polarization and  can be transited to circular polarization by
\begin{equation}
 	\begin{bmatrix}
	\hat{u}_{r}\hat{u}_{r} & \hat{u}_{r}\hat{u}_{l} \\
	\hat{u}_{l}\hat{u}_{r} & \hat{u}_{l}\hat{u}_{l}
	\end{bmatrix} = 
	\begin{bmatrix}
	\hat{u}_{r} \\
	\hat{u}_{l} 
	\end{bmatrix}
	\begin{bmatrix}
	\hat{u}_{r} \\
	\hat{u}_{l} 
	\end{bmatrix}^* = 
	\mathcal{T}_{rl \rightarrow hv}
	\begin{bmatrix}
	\hat{u}_{h}\hat{u}_{h} & \hat{u}_{h}\hat{u}_{v} \\
	\hat{u}_{v}\hat{u}_{h} & \hat{u}_{v}\hat{u}_{v}
	\end{bmatrix}
	\mathcal{T}_{rl \rightarrow hv}^*
\end{equation}

Where $\hat{u}$ is the coordinate, the subscripts $h$ and $v$ represent the horizontal and vertical plane, and the subscripts $r$ and $l$ represent the RHCP and LHCP plane. Then the Fresnel reflection coefficients of the circular polarization are

\begin{equation}
	\begin{bmatrix}
	\mathcal{R}_{rr} & \mathcal{R}_{rl} \\
	\mathcal{R}_{lr} & \mathcal{R}_{ll}
	\end{bmatrix} = 
	\frac{1}{2}
	\begin{bmatrix}
	\mathcal{R}_{hh} + \mathcal{R}_{vv} & \mathcal{R}_{hh} - \mathcal{R}_{vv} \\
	\mathcal{R}_{hh} - \mathcal{R}_{vv} & \mathcal{R}_{hh} + \mathcal{R}_{vv}		\end{bmatrix}
\end{equation}




% Note that the IEEE does not put floats in the very first column
% - or typically anywhere on the first page for that matter. Also,
% in-text middle ("here") positioning is typically not used, but it
% is allowed and encouraged for Computer Society conferences (but
% not Computer Society journals). Most IEEE journals/conferences use
% top floats exclusively. 
% Note that, LaTeX2e, unlike IEEE journals/conferences, places
% footnotes above bottom floats. This can be corrected via the
% \fnbelowfloat command of the stfloats package.

%\section{Dielectric constant of the soil}
%\label{sec:DC_soil}
Dieletric constant of soil mixture depends on the frequency, temperature, salinity of soil, the bulk soil density, and soil moisture \cite{Peplinski:1995,Peplinski_correct:1995}.

\begin{equation} \label{eq: soil_dielectric_main}
\begin{split}
		 { \varepsilon_{soil} } &=  { \varepsilon'_{soil} } -j { \varepsilon''_{soil} } \\
    	 { \varepsilon'_{soil} } &= [1 + \frac{\rho_b}{\rho_{ss}} ( {\varepsilon_{s}^{\alpha}}-1) + (m_v^{\beta'}\varepsilon_{fw}^{\prime\alpha}-m_v]^{1/\alpha} \\
         { \varepsilon''_{soil} } &=m_v^{\beta"}\varepsilon_{fw}^{''\alpha}
\end{split}
\end{equation}

Where $\varepsilon_{soil}$ is the dielectric constant of soil, $\rho_b$ is the bulk density, $\rho_{ss}$ is the density of the solid soil material, $\varepsilon'_{fw}$ and $\varepsilon''_{fw}$ are the real and imaginary part of the dielectric constant of the free water, and $\alpha=0.65$ and $\beta$ are adjustable parameters. The adjustable parameters $\beta$ is a function of soil texture as follows,

\begin{equation}
\begin{split}
	\beta' &= 1.2748-0.519Soil_{sand}-0.152Soil_{clay} \\
    \beta'' &=1.33797-0.603Soil_{sand}-0.166Soil_{clay}
 \end{split}
\end{equation}
where $S$ and $C$ represent the mass fractions of sand and clay, respectively. The bulk density is also a function of soil texture \cite{Saxton:1986}.

\begin{equation} \label{eq: bulk_density}
\begin{split}
	 \rho_b &= (1-\Psi)\rho_b \\
	  \Psi &= 0.332 - 7.251 \cdot 10^{-4} Soil_{sand} + 0.1276 log_{10}(Soil_{clay})
\end{split}
\end{equation}

$\varepsilon_{s} $ is the dielectric constant of the solid soil is given by
\begin{equation}
	\varepsilon_s = (1.01+0.44\rho_s)^2-0.062
\end{equation}
The dielectric constant of free water $\varepsilon'_{fw}$ and $\varepsilon''_{fw}$ are expressed as the follows,
\begin{equation}
\begin{split}
	\varepsilon'_{fw} &=\varepsilon_{w\infty} + \frac{\varepsilon_{w0}-\varepsilon_{w\infty}}{1+(2 \pi f \tau_w)^2} \\
    \varepsilon''_{fw} &=\frac{2 \pi f \tau_w(\varepsilon_{w0}-\varepsilon_{w\infty})}{1+(2 \pi f \tau_w)^2} + \frac{\sigma_{eff}}{2 \pi \varepsilon_0 f} \frac{\rho_s-\rho_b}{\rho_s m_v} 
\end{split}
\end{equation}
where $\varepsilon_0$ is the permittivity of free space, $\tau_w$ represents the relaxation time for water, $f$ is the frequency in Hz, $\varepsilon_{w\infty}$ = 4.9 is the high-frequency limit of $\varepsilon'_{fw}$, and $m_v$ is the Volumetric soil moisture. The effective conductivity $\sigma_{eff}$ can be estimated as
\begin{equation}
	\sigma_{eff} = 0.0467+0.2204\rho_b-0.4111Soil_{sand}+0.6614Soil_{clay}
\end{equation}


\begin{table}[ht]
\centering
\begin{tabular}  {|c|c|c|c|c|c|}
	\hline
     Polynomial &1&2 &3&4&5\\
    \hline
   P-Band & 372.81 & -1200.41 &	1709.41 & -1390.75 & 710.29\\
     \hline
     Polynomial &6&7&8&9 &10\\
    \hline
   P-Band & -232.89 &	49.53 &	-5.68 & 0.84 & -0.04 \\
    \hline
\end{tabular}
\caption{The coefficients of a polynomial of degree 9}
\label{Table:Polynomial_fitting}
\end{table}


\bf This is just a derivation of Newton's method ... \rm
\begin{eqnarray}
f_{Re}(\Gamma, \phi)=&\frac{(\sqrt{I_E}+\sqrt{I_S}\Gamma)R_a(\tau^s_{RD})+(\sqrt{\Gamma} R_a(0)+\sqrt{I_S I_E\Gamma} R_a(2\tau^s_{RD}))cos\phi}                              
                   {(1 + I_S)\Gamma)R_a(0)+2\sqrt{I_S\Gamma} R_a(\tau^s_{RD})cos\phi}   \\
f_{Im}(\Gamma, \phi)=&\frac{(\sqrt{\Gamma} R_a(0)-\sqrt{I_S I_E\Gamma} R_a(2\tau^s_{RD}))sin\phi }                             
                   {(1 + I_S)\Gamma)R_a(0)+2\sqrt{I_S\Gamma} R_a(\tau^s_{RD})cos\phi}   
    \label{Eq: Gamma_estimation_approx2}
\end{eqnarray}
The jacobian matrix of Equation \ref{Eq: Gamma_estimation_approx2} is as the following matrix, 
\begin{equation}
    J(\Gamma, \phi) =
    \begin{bmatrix}
        \frac{\partial f_{Re}}{\partial \Gamma}      &  \frac{\partial f_{Re}}{\partial \phi}  \\
        \frac{\partial f_{Im}}{\partial \Gamma}      &  \frac{\partial f_{Im}}{\partial \phi}  
    \end{bmatrix} \label{Eq: Jacobian}
\end{equation}

The reflectivity can be estimated from Equation \ref{Eq: Gamma_estimation_approx2} by the iteration. At first, the initial solutions $(\Gamma^{(0)}, \phi^{(0)})$ are substituted into \ref{Eq: Gamma_estimation_approx2}, the difference between the measurement  and estimation is  
\begin{equation}
\begin{bmatrix} \Delta f_{Re}^{(0)} \\ \Delta f_{Im}^{(0)} \end{bmatrix} = 
\begin{bmatrix} f_{Re}(\Gamma, \phi) -  f_{Re}(\hat{\Gamma}^{(0)}, \hat{\phi}^{(0)})\\ f_{Im}(\Gamma, \phi) -  f_{Im}(\hat{\Gamma}^{(0)}, \hat{\phi}^{(0)})\end{bmatrix}
\label{Eq: LSQ1}
\end{equation}
The solutions of reflectivity and phase can be updated as
\begin{align}
 \begin{bmatrix} \Delta \Gamma^{(0)} \\ \Delta \phi^{(0)} \end{bmatrix} &=
 \begin{bmatrix}
        \frac{\partial f_{Re}}{\partial \Gamma}      &  \frac{\partial f_{Re}}{\partial \phi}  \\
        \frac{\partial f_{Im}}{\partial \Gamma}      &  \frac{\partial f_{Im}}{\partial \phi}  
    \end{bmatrix} ^{-1} _{(\hat{\Gamma}^{(0)} ,\hat{\phi}^{(0)})}
    \begin{bmatrix} \Delta f_{Re}^{(0)} \\ \Delta f_{Im}^{(0)} \end{bmatrix} \\
\begin{bmatrix} \hat{\Gamma}^{(1)} \\\hat{\phi}^{(1)} \end{bmatrix}  &= 
\begin{bmatrix} \hat{\Gamma}^{(0)} \\ \hat{\phi}^{(0)} \end{bmatrix}  + \begin{bmatrix} \Delta \Gamma^{(0)} \\ \Delta \phi^{(0)} \end{bmatrix}
\label{Eq: LSQ2}
\end{align}
The final solutions are obtained until  the difference between the measurement  and estimation $(\Delta f_{Re}^{(n)} , \Delta f_{Im}^{(n)}) $  are smaller than tolerance.
\begin{align}
 \begin{bmatrix} \Delta \Gamma^{(n-1)} \\ \Delta \phi^{(n-1)} \end{bmatrix} &=
 \begin{bmatrix}
        \frac{\partial f_{Re}}{\partial \Gamma}      &  \frac{\partial f_{Re}}{\partial \phi}  \\
        \frac{\partial f_{Im}}{\partial \Gamma}      &  \frac{\partial f_{Im}}{\partial \phi}  
    \end{bmatrix} ^{-1} _{(\hat{\Gamma}^{(n-1)} ,\hat{\phi}^{(n-1)})}
    \begin{bmatrix} \Delta f_{Re}^{(n-1)} \\ \Delta f_{Im}^{(n-1)} \end{bmatrix} \\
\begin{bmatrix} \hat{\Gamma}^{(n)} \\ \hat{\phi}^{(n)} \end{bmatrix}  &= 
\begin{bmatrix} \hat{\Gamma}^{(n-1)} \\ \hat{\phi}^{(n-1)} \end{bmatrix}  + \begin{bmatrix} \Delta \Gamma^{(n-1)} \\ \Delta \phi^{(n-1} \end{bmatrix} \\
\begin{bmatrix} \Delta f_{Re}^{(n)} \\ \Delta f_{Im}^{(n)} \end{bmatrix} &= 
\begin{bmatrix} f_{Re}(\Gamma, \phi) -  f_{Re}(\Gamma^{(n-1)}, \phi^{(n-1)})\\ f_{Im}(\Gamma, \phi) -  f_{Im}(\Gamma^{(n-1)}, \phi^{(n-1)})\end{bmatrix}
\label{Eq: LQS3}
\end{align}


% use section* for acknowledgment
\section*{Acknowledgment}


The authors would like to thank...


% Can use something like this to put references on a page
% by themselves when using endfloat and the captionsoff option.
\ifCLASSOPTIONcaptionsoff
  \newpage
\fi



% trigger a \newpage just before the given reference
% number - used to balance the columns on the last page
% adjust value as needed - may need to be readjusted if
% the document is modified later
%\IEEEtriggeratref{8}
% The "triggered" command can be changed if desired:
%\IEEEtriggercmd{\enlargethispage{-5in}}

% references section

% can use a bibliography generated by BibTeX as a .bbl file
% BibTeX documentation can be easily obtained at:
% http://www.ctan.org/tex-archive/biblio/bibtex/contrib/doc/
% The IEEEtran BibTeX style support page is at:
% http://www.michaelshell.org/tex/ieeetran/bibtex/
%\bibliographystyle{IEEEtran}
% argument is your BibTeX string definitions and bibliography database(s)
%\bibliography{IEEEabrv,../bib/paper}
%
% <OR> manually copy in the resultant .bbl file
% set second argument of \begin to the number of references
% (used to reserve space for the reference number labels box)
\begin{thebibliography}{1}

\bibitem{Ulaby:1981}
F. T. Ulaby, R. K. Moore, and A. K. Fung, Microwave Remote Sensing: From theory to applications: Addison-Wesley Publishing Company, Advanced Book Program/World Science Division, 1981.

\bibitem{Saxton:1986}
K. E. Saxton, W. J. Rawls, J. S. Romberger, and R. I. Papendick, "Estimating Generalized Soil-water Characteristics from Texture1," Soil Sci. Soc. Am. J., vol. 50, pp. 1031-1036, 1986 1986.

\bibitem{Peplinski:1995}
N. R. Peplinski, F. T. Ulaby, and M. C. Dobson, "Dielectric properties of soils in the 0.3-1.3-GHz range," Geoscience and Remote Sensing, IEEE Transactions on, vol. 33, pp. 803-807, 1995.

\bibitem{Peplinski_correct:1995}
N. R. Peplinski, F. T. Ulaby, and M. C. Dobson, "Corrections to "Dielectric Properties of Soils in the 0.3-1.3-GHz Range"," Geoscience and Remote Sensing, IEEE Transactions on, vol. 33, p. 1340, 1995.

\bibitem{Jin:2011}
S. Jin, G. P. Feng, and S. Gleason, "Remote sensing using GNSS signals: Current status and future directions," Advances in Space Research, vol. 47, pp. 1645-1653, 5/17/ 2011.

\bibitem{Zavorotny:2010}
V. U. Zavorotny, K. M. Larson, J. J. Braun, E. E. Small, E. D. Gutmann, and A. L. Bilich, "A Physical Model for GPS Multipath Caused by Land Reflections: Toward Bare Soil Moisture Retrievals," Selected Topics in Applied Earth Observations and Remote Sensing, IEEE Journal of, vol. 3, pp. 100-110, 2010.

\bibitem{Matt:2014}
K. Rautiainen, J. Lemmetyinen, M. Schwank, A. Kontu, C. B. Ménard, C. Mätzler, et al., "Detection of soil freezing from L-band passive microwave observations," Remote Sensing of Environment, vol. 147, pp. 206-218, 5/5/ 2014.

\bibitem{Larson:2008}
K. M. Larson, E. E. Small, E. D. Gutmann, A. L. Bilich, J. J. Braun, and V. U. Zavorotny, "Use of GPS receivers as a soil moisture network for water cycle studies," Geophys. Res. Lett., vol. 35, p. L24405, 2008.

\bibitem{Shah:2011}
R. Shah, J. L. Garrison, and M. S. Grant, "Anisotropy in ocean scattering of bistatic radar using signals of opportunity," in Geoscience and Remote Sensing Symposium (IGARSS), 2011 IEEE International, 2011, pp. 4229-4232.

\bibitem{Kerr:2000}
Y. H. Kerr, J. Font, P. Waldteufel, and M. Berger, "The Soil Moisture and Ocean Salinity Mission - SMOS," in ESA Earth Observation Quarterly, ed, 2000, pp. 18-25.

\bibitem{Wang:2009}
L. Wang and J. J. Qu, "Satellite remote sensing applications for surface soil moisture monitoring: A review," Frontiers of Earth Science in China, vol. 3, pp. 237-247, 2009.

\bibitem{Entekhabi:2010}
D. Entekhabi, E. G. Njoku, P. E. O'Neill, K. H. Kellogg, W. T. Crow, W. N. Edelstein, et al., "The Soil Moisture Active Passive (SMAP) Mission," PROCEEDINGS OF THE IEEE, vol. 98, pp. 704-716, 2010.

\end{thebibliography}

% biography section
% 
% If you have an EPS/PDF photo (graphicx package needed) extra braces are
% needed around the contents of the optional argument to biography to prevent
% the LaTeX parser from getting confused when it sees the complicated
% \includegraphics command within an optional argument. (You could create
% your own custom macro containing the \includegraphics command to make things
% simpler here.)
%\begin{IEEEbiography}[{\includegraphics[width=1in,height=1.25in,clip,keepaspectratio]{mshell}}]{Michael Shell}
% or if you just want to reserve a space for a photo:

\begin{IEEEbiography}{Michael Shell}
Biography text here.
\end{IEEEbiography}

% if you will not have a photo at all:
\begin{IEEEbiographynophoto}{John Doe}
Biography text here.
\end{IEEEbiographynophoto}

% insert where needed to balance the two columns on the last page with
% biographies
%\newpage

\begin{IEEEbiographynophoto}{Jane Doe}
Biography text here.
\end{IEEEbiographynophoto}

% You can push biographies down or up by placing
% a \vfill before or after them. The appropriate
% use of \vfill depends on what kind of text is
% on the last page and whether or not the columns
% are being equalized.

%\vfill

% Can be used to pull up biographies so that the bottom of the last one
% is flush with the other column.
%\enlargethispage{-5in}



% that's all folks
\end{document}