\section{Uncertainty analysis}

Parameters of uncertainty analysis inlude antenna gains ($G_{S,D}, G_{S,R}, G_{E,D}, G_{E,R}$), channel gain ($G_1, G_2$), noise power spectra ($\sigma_1$), reflectivity ($\Gamma$), and phase ($\phi$).  First, (\ref{eqn:observation_eqn_general}) without noise residual  term is expressed as
\begin{equation}
g(x, \tilde{x}) = \Gamma_{12}(x, \tilde{x}) - f(\tilde{x})
\end{equation}
Where $x=\{G_{S,D}, G_{S,R}, G_{E,D}, G_{E,R}, G_1, G_2, \sigma_1, \Gamma, \phi\}$ and $\tilde{x}=\{\tilde{G}_{S,D}, \tilde{G}_{S,R}, \tilde{G}_{E,D}, \tilde{G}_{E,R}, \tilde{G}_1, \tilde{G}_2, \tilde{\sigma}_1, \tilde{\Gamma}, \tilde{\phi}\}$ are the set of the parameters for the measurements and calibration terms. Once the claibration terms $\tilde{x}$ exactly matches $x$, $g(\Gamma, \phi, x, \tilde{x}) = 0$. The full expression of $g(\Gamma,\phi)$ is
\begin{equation}
g(x,\tilde{x}) = \sqrt{\frac{\tilde{G}_1}{\tilde{G}_2}} \sqrt{\frac{\tilde{G}_{S,D}}{\tilde{G}_{E,R}}}  
\frac{R_{1,2}(\tau^s_{RD})}{R_{1,1}(0)-\tilde{G}_1 \sigma^2_1} -
 \frac
{\sqrt{\tilde{I}_E}+ \sqrt{\tilde{I}_S} \tilde{\Gamma})R_a(\tau_{RD})+
\sqrt{\tilde{\Gamma}}R_a(0)e^{j\tilde{\phi}}+
\sqrt{\tilde{I}_E \tilde{I}_S\tilde{\Gamma}}R_a(2\tau_{RD})e^{-j \tilde{\phi} } )
}
{(1+\tilde{I}_S \tilde{\Gamma})R_a(0)+
2 \sqrt{\tilde{I}_S\tilde{\Gamma}}(\mathcal{Re} (R_a(\tau_{RD})) )cos\tilde{\phi}
}                            
\end{equation}
The total derivative of $g(x,\tilde{x})$ is
\begin{align}
dg(x,\tilde{x})&=\sum_{x_i \in \tilde{x}} \frac{\partial(g)}{\partial(x_i)} dx_i \\
\Rightarrow dg(x,\tilde{x})&=\sum_{x_i \in \tilde{x}} (\frac{\partial(\Gamma_{12})}{\partial(x_i)} dx_i + \frac{\partial(f)}{\partial(x_i)}dx_i)
\end{align}
The first term is the total derivative of $\Gamma_{12}$, and the following equations shows the partial derivative of different parameters.
\begin{equation}
\tilde{G}_1 :
\sqrt{\frac{\tilde{G}_2}{\tilde{G}_1}}
\sqrt{\frac{\tilde{G}_{S,D}}{\tilde{G}_{E,R}}}
\frac{R_{1,2}(\tau_{RD}) (R_{1,1}(0) + \tilde{G}_1 \sigma_1^2)}
{2\tilde{G}_2(R_{1,1}(0) - \tilde{G}_1 \sigma_1^2)^2 )} 
\end{equation}
\begin{equation}
\tilde{G}_2 :
\sqrt{\frac{\tilde{G}_2}{\tilde{G}_1}}
\sqrt{\frac{\tilde{G}_{S,D}}{\tilde{G}_{E,R}}}
-\frac{R_{1,2}(\tau_{RD})}
{2\tilde{G}_2^2(R_{1,1}(0) - \tilde{G}_1 \sigma_1^2) )} 
\end{equation}
\begin{equation}
\tilde{G}_{S,D}:
 \sqrt{\frac{\tilde{G}_1}{\tilde{G}_2}}
 \sqrt{\frac{\tilde{G}_{E,R}}{\tilde{G}_{S,D}}}  
\frac{R_{1,2}(\tau^s_{RD})}{2 G_{E,R} (R_{1,1}(0)-\tilde{G}_1 \sigma^2_1)} 
\end{equation}
\begin{equation}
\tilde{G}_{S,D}:
 -\sqrt{\frac{\tilde{G}_1}{\tilde{G}_2}}
 \sqrt{\frac{\tilde{G}_{E,R}}{\tilde{G}_{S,D}}}  
\frac{R_{1,2}(\tau^s_{RD})}{2 G_{E,R}^2 (R_{1,1}(0)-\tilde{G}_1 \sigma^2_1)} 
\end{equation}\begin{equation}
\tilde{\sigma}_1:
\sqrt{\frac{\tilde{G}_1}{\tilde{G}_2}} \sqrt{\frac{\tilde{G}_{S,D}}{\tilde{G}_{E,R}}}  
\frac{2 \tilde{G}_1 \tilde{\sigma}_1 R_{1,2}(\tau^s_{RD})}{(R_{1,1}(0)-\tilde{G}_1 \sigma^2_1)^2} 
\end{equation}